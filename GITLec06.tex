\sektion{6}{More CST}

There is a classification of complex reflection groups. Of course, the usual reflection groups are Coxeter groups (1934). Shephard and Todd in 1954 classified all finite complex reflection groups. \anton{add refs to bibliography and put this comment some place else (after complete proof of CST?)}

Recall that for a finite group $G$ (with $\mathrm{char}(k)\nmid |G|$) acting linearly on an $n$-dimensional vector space $V=\specm R$, Moilen's formula tells us that
\[
 P_{R^G}(t)=\frac{1}{|G|}\sum_{g\in G} \frac{1}{\det(1-gt)}.
\]
If the eigenvalues of the action of $g$ are $a_1,\dots, a_n$, then $\frac{1}{\det(1-gt)}=\frac{1}{(1-a_1t)\cdots (1-a_nt)}$. The identity element of $G$ is the only term in the sum that contributes $\frac{1}{(1-t)^n}$, and the only terms that contribute a multiple of $\frac{1}{(1-t)^{n-1}}$ are reflections, the $g$ which have a single non-trivial eigenvalue. So by expanding $P_{R^G}(t)$ as a Laurent series around $t=1$, we can extract $|G|$ and the number of reflections in $G$.
\begin{lemma}\label{lec6Lem:ord+refls}
 In the Laurent series expansion $P_{R^G}(t)= \sum_{i=-n}^\infty c_i(1-t)^i$, $c_{-n}=\frac{1}{|G|}$ and $c_{1-n}=\frac{1}{2|G|}|S_G|$, where $S_G\subseteq G$ is the set of reflections in $G$.
\end{lemma}
\begin{proof}
 As we've already noted, the identity element is the only term in Moilen's formula that contributes a pole of order $n$, so the lowest order term in the Laurent series expansion will be $\frac{1}{|G|}(1-t)^{-n}$.

 We've also seen that the only elements $g\in G$ that contribute a pole of order $n-1$ are the reflections. If the only non-trivial eigenvalue of $g$ is $\e$, then the coefficient of $(1-t)^{1-n}$ in the Laurent series expansion of $\frac{1}{\det(1-tg)}$ is
 \[
  \der{}{t}\biggl(\frac{-(1-t)^n}{(1-t)^{n-1}(1-\e t)}\biggr)\biggr|_{t=1} = \frac{(1-\e t)-\e(1-t)}{(1-\e t)^2}\biggr|_{t=1} = \frac{1}{1-\e}.
 \]
 If $g$ is a reflection of order $2$, then $\e=-1$, so it contributes $\frac 12$ to the coefficient of $(1-t)^{n-1}$. Otherwise, $g\neq g^{-1}$, and together these two reflections contribute
 \[
  \frac{1}{1-\e} + \frac{1}{1-\e^{-1}} = \frac{1-\e^{-1}+1-\e}{1-\e-\e^{-1}+\e\e^{-1}} = 1
 \]
 to the coefficent of $(1-t)^{1-n}$, so we may think of $g$ and $g^{-1}$ as each contributing $\frac 12$. Adding these contributions up and multiplying by the $\frac{1}{|G|}$ from Moilen's formula, we get $c_{1-n}=\frac{1}{2|G|}|S_G|$.
\end{proof}
\begin{proposition}\label{lec6Prop:almost_CST}
 If  
 \[
  P_{R^G}(t)= \frac{1}{(1-t^{d_1})\cdots (1-t^{d_n})}
 \]
 then $|G|=\prod_i d_i$ and $|S_G|=\sum_i (d_i-1)$.
\end{proposition}
\begin{proof}
 Let the Laurent series expansion be
 \[
  \frac{1}{(1-t^{d_1})\cdots (1-t^{d_n})} = P_{R^G}(t) = \sum_{i=-n}^\infty c_i (1-t)^i.
 \]
 Then we compute
 \begin{align*}
  c_{-n} &= \frac{(1-t)^n}{(1-t^{d_1})\cdots (1-t^{d_n})}\biggr|_{t=1} =\frac{1}{\prod_i(1+t+\cdots +t^{d_i-1})} = \frac{1}{\prod_i d_i}\\
  c_{1-n} &= \der{}{t}\biggl(\frac{-(1-t)^n}{(1-t^{d_1})\cdots (1-t^{d_n})}\biggr)\biggr|_{t=1}= \der{}{t}\biggl(-\prod_i\frac{1}{1+t+\cdots +t^{d_i-1}}\biggr)\biggr|_{t=1}\\
  &= \sum_i \frac{d_i(d_i-1)}{2d_i^2}\cdot \frac{1}{\prod_{j\neq i} d_j} = \frac{1}{2\prod_j d_j} \sum_i (d_i-1).
 \end{align*}
 Now the result follows from Lemma \ref{lec6Lem:ord+refls}.
\end{proof}
\begin{proposition}\label{lec6Prop:other_dir_CST}
 If
 \[
  P_{R^G}(t)= \frac{1}{(1-t^{d_1})\cdots (1-t^{d_n})}
 \]
 then $G$ is generated by reflections.
\end{proposition}
\begin{proof}
 \anton{For some reason} $R^G$ must be a polynomial algebra \anton{this should probably just be part of the hypothesis of the proposition}. Suppose $R^G=k[f_1,\dots, f_n]$ with $\deg f_i=d_i$. Let $H\subseteq G$ be the subgroup generated by all the reflections in $G$. By Proposition \ref{lec5:one_dir_CST}, $R^H=k[h_1,\dots, h_n]$ and $P_{R^H}(t)=\frac{1}{(1-t^{e_1})\dots (1-t^{e_n})}$. We may assume that $d_1\le \cdots \le d_n$ and $e_1\le \cdots \le e_n$.

 We claim that $e_i\le d_i$. To see this, suppose $d_p<e_p$ for some $p$. By degree considerations, and using that $R^G\subseteq R^H\subseteq R$, we have that $f_i= Q_i(h_1,\dots, h_{p-1})$ for some polynomials $Q_i$ and all $i\le p$. But $Q_1,\dots, Q_p\in k[h_1,\dots, h_{p-1}]$ must satisfy an algebraic relation since the Krull dimension of $k[h_1,\dots, h_{p-1}]$ is $p-1$. This contradicts the algebraic independence of the $f_i$.

 On the other hand, the number of reflections in $G$ and $H$ must be equal, so $\sum_i (d_i-1)=\sum_i (e_i-1)$. Combining this with the inequality $e_i\le d_i$, we must have $e_i=d_i$ for all $i$. Then by Lemma \ref{lec6Lem:ord+refls}, we have
 \[
  |H| = \prod_i e_i =\prod_i d_i=|G|
 \]
 so $G=H$, so $G$ is generated by reflections.
\end{proof}
Combining Lemma \ref{lec6Lem:ord+refls} with Propositions \ref{lec5:one_dir_CST} and \ref{lec6Prop:other_dir_CST}, we get the following theorem.
\begin{theorem}[Chevalley-Shephard-Todd]
 Suppose $G\subseteq GL(V)$. The ring $k[V]^G$ is isomorphic to a polynomial ring if and only if $G$ is generated by reflections. If this is the case, with $k[V]^G=k[f_1,\dots, f_n]$ and $\deg f_i=d_i$, then $|G|=\prod_i d_i$ and $|S_G|=\sum_i (d_i-1)$.
\end{theorem}
Question: do the other symmetric functions in the $d_i$ give you more information? Answer: I don't see how. Maybe if you write out more terms in the Laurent series expansion, you'll get something.

\begin{lemma}\label{lec6Lem:independence_lifts}
 For any linearly independent set $\{y_1+I,\dots, y_k+I\}$ in $R/I$ (where $I$ is the ideal generated by $R^G_{>0}$), the set $\{y_1,\dots, y_k\}$ is $R^G$-linearly independent in $R$.
\end{lemma}
\begin{proof}
 We use induction on $k$. If $k=1$, then the result is clear since $R$ has no zero-divisors.

 Suppose you have a relation $h_1y_1+\cdots+h_ky_k=0$ for some $h_i\in R^G$. By assumption, $y_1\not\in I$, so by Lemma \ref{lec5Lem:R_G}, we get that $h_1\in R^Gh_2+\cdots+R^G h_k$, say $h_1=u_2h_2+\cdots u_kh_k$. Substituting into the previous relation, we get the relation
 \[
  h_2(y_2-u_2y_1)+\cdots + h_k(y_k-u_ky_1)=0.
 \]
 Since the $u_i\in R^G$ reduce to scalars in $R/I$, $\{y_2-u_2y_1+I, \dots, y_k-u_ky_1+I\}$ is a linearly independent set in $R/I$, so we have $h_2=\cdots =h_k=0$ by the inductive hypothesis, and then $h_1=0$ by the base case.
\end{proof}

\begin{proposition}
 If $G$ is generated by reflections, then $R$ is a free $R^G$-module.
\end{proposition}
We already discussed that we know the rank because we know that $K$ is a degree $|G|$ extension of $K^G$.
\begin{proof}
 To choose a set of generators, consider the ideal $I=R^G_{>0}\subseteq R$. Choose homogeneous elements $y_1,\dots, y_\ell\in R$ such that $y_1+I,\dots, y_\ell+I$ form a basis for $R/I$. By induction on degree, the $y_i$ generate $R$ as an $R^G$-module. So we need only to show independence. \anton{they generate by the graded version of Nakayama's lemma: for any graded ideal $I$ and any finitely generated module $M$, $M=IM\Rightarrow M=0$ and generators for $M/IM$ lift to generators of $M$.}
 
 Independence follows from Lemma \ref{lec6Lem:independence_lifts}
\end{proof}
More generally, a finite extension of polynomial algebras is free. \anton{the same proof doesn't work ... can we get it to work?}

\subsektion{Semi-invariants}

Suppose $G$ acts on $X$. Suppose I have a character $\chi\colon G\to k^\times$. A function satisfying the condition $f(gx)=\chi(g)f(x)$ is called \emph{semi-invariant}. So even though $f$ is not invariant, the corresponding line $kf(x)$ is invariant.

Suppose you want some rational function $f/g$ to be invariant, it suffices for $f$ and $g$ to be semi-invariant with the same character.

We are going to construct semi-invariants for groups generated by reflections. Look at the set $S_G$ of all reflections. Clearly $G$ acts on $S_G$ by conjugation. For each $s$, we have the associated hyperplane $H_s$ and a linear functional $\ell_s=H^\perp\subseteq V^*$. Decompose $S_G$ as a union of orbits. $S_G=O_1\cup \cdots \cup O_k$. Then define
\[
 f_{O_i}=\prod_{s\in O_i} \ell_s.
\]
We claim that this is semi-invariant. This is easy to check using the identity
\[
 g(\ell_s)= \ell_{gsg^{-1}}.
\]
(up to scalar).
\begin{proposition}
 Any semi-invariant can be written uniquely in the form $f_{O_1}^{a_1}\cdots f_{O_k}^{a_k}f_0$, where $f_0\in R^G$, and $0\le a_i\le \mathrm{ord}(s)$ where $s\in O_i$.
\end{proposition}
\begin{proof}
 First, induction on degree. Pick a semi-invariant $f$. If it is not invariant, pick some $s\in S_G$ such that $s(f)=\e f$ for $\e\neq 1$. Then I claim that $\ell_s|f$ because $\ell_s$ divides $f-s(f)=(1-\e)f$. Then $\ell_{s'}|f$ for any $s'$ in the orbit of $s$. Then procede by induction.
\end{proof}









