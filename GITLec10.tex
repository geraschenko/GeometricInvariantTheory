\sektion{10}{Classifying Reductive Groups, Part II}

Statements we proved last time:
\begin{enumerate}
 \item If $G=G_s$ and $G$ abelian, then $G=\Ga\times G_0$ where $G_0$ is a torus and $\Ga$ is a finite group with $\mathrm{char}(k)\nmid |\Ga|$.
 \item $s\in G_s$, then $\lie(C_G(s))=C_\g(s)$. From which we get the corollary: If $C_\g(s)=\g$ and $G$ is connected, then $s\in Z(G)$.
 \item If $G$ is reductive, then $Z(G)=Z(G)_s$.
 \item If $G$ is reductive and $\mathrm{char}(k)>0$, then $\g=\g_s$.
\end{enumerate}
Then we were in the middle of the following
\begin{proposition}
 Suppose $\g=\g_s$ and $G$ is connected. Then $G$ is abelian (and therefore a torus).
\end{proposition}
So we get a description of reductive groups in characteristic $p$.
\begin{proof}
 We were doing induction on $\dim G$ and $\dim V$, where $G\subseteq GL(V)$. By extending our ground field, we showed that the elements of infinite order is a dense set.
 
 Let $g\in G$ be of infinite order, and let $H=[C_G(g)]_0$ (connected component). Since $g$ is of infinite order, $\dim H\ge 1$ (since it isn't finite since it contains the powers of $g$). If for any infinite order element $g$ we get $H=G$, then we are done since elements of infinite order form a dense set. So we may assume $H\neq G$. Then by induction on $\dim G$, $H$ is abelian and hence is a torus.
 
 Since $H$ is a torus, we get the decomposition $V=\bigoplus_{\chi\in P} V_\chi$, where $P\subseteq H^\vee$ and $V_\chi=\{v\in V| hv=\chi(h)v\}$. Similarly, we get $\g=\bigoplus_{\chi\in Q} \g_\chi$ for some $Q\subseteq H^\vee$. Now it is easy to check that $\g_\chi V_\eta\subseteq V_{\chi\eta}$. This tells me that $\g_\chi^n V_\eta\subseteq V_{\chi^n\eta}$. But $P$ is finite, so if $\chi\neq 1$, we must have that $\g_\chi$ is nilpotent. But since $\g=\g_s$, there are no nilpotent elements, so $\g_\chi=0$ for $\chi\neq 1$. Thus, the adjoint action of $H$ on $\g$ is trivial. So $H\subseteq Z(G)$.
 
 Suppose $|P|>1$. Then since the action of $G$ commutes with the action of $H$, each $V_\chi$ is $G$-invariant. We have projections $\pi\colon G\to G_\chi\subseteq GL(V_\chi)$. But by induction on $\dim V$, each $G_\chi$ is abelian. So $G$ is abelian.
 
 So we may assume $|P|=1$. So $H$ consists only of scalar matrices. Then $\tilde G=[G\cap SL(V)]_0$ has dimension smaller than $G$, and $G$ is generated by $\tilde G$ and $H$. By induction on $\dim G$, $\tilde G$ is abelian.
\end{proof}
\begin{corollary}
 If $G$ is a reductive connected group and $\mathrm{char}(k)=p>0$, then $G$ is a torus.
\end{corollary}
\begin{theorem}
 $G$ is reductive in characteristic $\mathrm{char}(k)=p>0$ if and only if $G_0$ is a torus and $p\nmid |G/G_0|$.
\end{theorem}
\begin{proposition}
 $G$ is reductive if and only if $G_0$ is reductive and $G/G_0$ is reductive.
\end{proposition}
\begin{proof}
 Suppose $G$ is reductive, and let $V$ be a representation of the quotient group $G/G_0$. It lifts to a representation of $G$, and the representation must split completely because $G$ is reductive.
 
 Now suppose $V$ is a not completely reducible representation of $G_0$, then $W=k[G]\otimes_{k[G_0]}V$ is a finite-dimensional representation of $G$ which is not completely reducible. \anton{check for yourself}
 
 Now suppose $G_0$ and $G/G_0$ are reductive. Recall that it is enough to show that $V=V^G\oplus W$. We have that $V=V^{G_0}\oplus W'$ since $G_0$ is reductive, and we have $V^{G_0}=V^G\oplus W''$ as a representation of $G/G_0$. \anton{it is easy to check that $W'$ is invariant under the action of $G$}
\end{proof}
By the way, we had a little question about why it was enough to check something on finite-dimensional representations. One of the criteria was that if $V\to W$ is surjective, then so is $V^G\to W^G$. Since every representation is a union of finite-dimensional representations, we get what we want.

\begin{remark}[Restricted Lie algebras]
 Suppose $\mathrm{char}(k)=p>0$. You can check that if $D\in \mathrm{Der}(k[G])$, then $D^p$ is again a derivation (follows from the binomial theorem). So we have a homomorphism of derivations $D\mapsto D^p$ (it plays well with the bracket). Left invariant derivations are sent to left invariant derivations, so we get a homomorphism $\g\to \g$, denoted by $x\mapsto x^{(p)}$. Such a Lie algebra (there are some axioms) is called a \emph{restricted Lie algebra}.
\end{remark}
\begin{example}
 If $G=k$, then $k[G]=k[t]$ and $\g=k\cdot\pder{}{t}$. We have that $(\pder{}{t})^p=0$, so our map is $x^{(p)}=0$.
 
 On the other hand, if $G=k^\times$, then $k[G]=k[t,t^{-1}]$ and $\g=k\cdot t\pder{}{t}$. Now we have $(t\pder{}{t})^p=t\pder{}{t}$.
 
 So even though these algebras are isomorphic, they are distinguishable as restricted Lie algebras.
\end{example}

\subsektion{Reductive groups in characteristic zero}

\begin{definition}
 A Lie algebra $\g$ is \emph{simple} if it is not abelian and $\g$ has no non-trivial proper ideals. We say $\g$ is \emph{semi-simple} if it is a direct sum of simple Lie algebras.
\end{definition}
The point of 261 is that semi-simple algebras are possible to classify in characteristic 0 over an algebraically closed field. In characteristic $p$, I think it is also possible, but there are more of them and it is more difficult.
\begin{definition}
 A group $G$ is \emph{simple} (resp.~\emph{semi-simple}) if its Lie algebra $\g$ is.
\end{definition}
\begin{theorem}[Weyl's Theorem]
 Every finite-dimensional representation over a semi-simple Lie algebra is completely reducible.
\end{theorem}
\begin{theorem}
 Any semi-simple algebraic group $G$ is reductive in characteristic zero.
\end{theorem}
\begin{proof}
 It is sufficent to check for $G$ connected since finite groups are reductive. We showed that in this situation $W\subseteq V$ is an invariant subspace if and only if $W\subseteq V$ is $\g$-invariant.
\end{proof}
\begin{remark}
 If a module is completely reducible over an algebraic closure, then it is completely reducible over the original field, so you don't need algebraically closed.
\end{remark}

Now assume $G$ is reductive. Then the adjoint representation $\Ad_G$ is completely reducible, so $\ad_\g$ is completely reducible. Thus, we have $\g=\g_1\oplus\cdots\oplus \g_k$. Each $\g_i$ is a submodule (ideal) with no proper non-trivial ideals, so each $\g_i$ is either simple or 1-dimensional. Thus, we may rewrite $\g=\g_1\oplus\cdots \oplus \g_r \oplus Z(\g)=\g^{ss}\oplus Z(\g)$, where each $\g_i$ is semi-simple and $Z(\g)$ is the center. Note that $[\g,\g]=\g^{ss}$.

Fact: \anton{exercise} If $G$ is a connected algebraic group, then $G'=[G,G]=\{ghg^{-1}h^{-1}|g,h\in G\}$ is a closed connected subgroup with Lie algebra $\lie G'=[\g,\g]$.

Now assume $G$ is connected and reductive. Then $G'\times Z(G)_0\to G$ is surjective. By fact (3), $Z(G)_0$ is a torus. So $G$ is a quotient of a product of a semi-simple group $G'$ with a torus by a finite central subgroup.

Typical example: $GL_n=(SL_n\times k^\times)/\mu_n$.

\begin{theorem}
 If $\mathrm{char}(k)=0$, then $G$ is reductive if and only if $G$ is a quotient of $G^{ss}\times T$ ($G^{ss}$ semi-simple and $T$ a torus) by some finite subgroup.
\end{theorem}
It's clear that any such quotient is reductive. I've implicitly used the following.
\begin{exercise}
 If $G_1$ and $G_2$ are reductive, then $G_1\times G_2$ is reductive.
\end{exercise}
