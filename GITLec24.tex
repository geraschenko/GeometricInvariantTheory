\sektion{24}{Lecture 24}

We were considering the action of $T$ on $\AA^n$ by the characters $\chi_1$,\dots, $\chi_n$. We fix a character $\chi$, which gives us a linearization of the structure sheaf. We defined $M=\{a\in \ZZ^n| \sum a_i\chi_i=0\}$. The quotient is given by $Y=\AA^n\quot_{L_\chi}T = \proj k[S]$. We chose monomial generators $f_1,\dots, f_p$ for $k[S]_{>0}$. Each $f_i$ is of the form $x^{m_i}$, and we defined $J_i=\supp m_i$.

We defined polyhedral rational cones $\sigma_i^\vee = \{a\in \RR^n|a_j\ge 0$ for all $j\in J_i\}$. $Y=\bigcup Y_i$, where $\O(Y_i)=R_i=k[\sigma_i^\vee\cap M]$.

I can think of $T$ as sitting inside the larger torus $U\cong (\AA\setminus \{0\})^n$. $U$ has a natural action on $\AA^n$, given by $(s_1,\dots, s_n)\cdot (x_1,\dots, x_n)=(s_1x_1,\dots, s_nx_n)$, so we get a natural map $T\to U$. We get another torus $Q=(U/T)_0$ (connected component of the identity, so it's really a torus). We have that $M\cong Q^\vee$, and $k[M]=k[Q]\supset R_i$ \anton{LHS is monoid algebra, and RHS is coordinate ring}. We have the comultiplication $\Delta\colon k[M]\to k[M]\otimes k[M]$ comming from the group structure on $Q$. On the other hand, we can restrict to $R_i$ to get $\Delta\colon R_i\to k[M]\otimes R_i$. So $Q$ acts on each $Y_i$, and they glue together equivariantly, so we get an action of $Q$ on $Y$. This isn't too surprising; we quotiented by a subgroup, so the quotient group should still act. Moreover, $Y$ has one open dense orbit.

Consider $(\ZZ^n)^*\subseteq (\RR^n)^*$. The natural embedding $M\hookrightarrow \ZZ^n$ gives the dual map $(\ZZ^n)^*\to M^*$. Let $M^*_\RR=M^*\otimes_\ZZ \RR$. Then we get a map $(\RR^n)^*\to M^*_\RR$.

We introduce a basis $\{\e_1,\dots, \e_n\}$ for $(\ZZ^n)^*$. By $\bar\e_i$, we denote the images of the $\e_i$ in $M^*_\RR$. We have the collection $J_1,\dots, J_p$. We say that $\sigma_i$ is the cone generated by $\{\bar\e_j| j\not\in J_i\}$. We define $\Sigma$ to be the collection $\sigma_1,\dots, \sigma_p$. It is clear (almost by definition) that $\sigma_i$ is dual to $\sigma_i^\vee\cap M$.
\begin{lemma}
 $\sigma_k\cap\sigma_\ell$ is a face of $\sigma_k$ and $\sigma_\ell$.
\end{lemma}
A face is a subcone such that there is a linear functional which is zero on the subcone and positive on the rest of the cone.
\begin{proof}
 We have that $\O(Y_k\cap Y_\ell)=R_k(x^{-c}]=R_\ell[x^{c}]$ for $c\in M$, with $\supp c\subseteq J_k\cup J_\ell$. Then $\<c,\sigma_k\>\subseteq \RR_{\ge 0}$ and $\<c,\sigma_\ell\>\subseteq \RR_{\le 0}$. It follows that $\<c,\sigma_k\cap \sigma_\ell\>=0$.
\end{proof}
\begin{example}
 $\PP^{n-1}=\AA^n\quot k^\times$. In this case, we have $J_1,\dots, J_n$, where $J_i$ are one-element sets (since the generators for $k[S]_{>0}$ are the $x_i$). We have one relation $\e_1+\cdots +\e_n=0$. $\sigma_i$ is generated by all the $\e_j$ for which $j\neq i$. In the case $n=3$, we get a partition of the plane by three cones.
\end{example}
\begin{example}
 $t\cdot (x_1,x_2,x_3, x_4)=(tx_1,tx_2,t^{-1}x_3,t^{-1}x_4)$. Now I get $\bar\e_1+\bar\e_2=\bar\e_3+\bar\e_4$. The fan will be in three dimensions. It will be the cone on a square, with $\e_1$ and $\e_2$ opposite corners of the square.
 
 In the case of $Y_0$, I just get $J_1=\varnothing$, so I get a single cone, which is not simplicial. This gives a quotient with a singularity.
 
 In the case of $Y_+$, I get $J_1=\{1\}$ and $J_2=\{2\}$, so I break my square cone into two simplicial cones (one excluding $\e_1$ and the other excluding $\e_2$). Similarly, $Y_-$ breaks the square cone into two simplicial cones, but the other way.
\end{example}
\begin{example}
 Recall the action of the 5-dimensional torus on the 6-dimensional space $\matx{x_1&x_2&x_3\\ y_1&y_2&y_3}$. In this case, the natural basis is $\e_1, \e_2, \e_2$ (corresponding to the $x$'s) and $\delta_1,\delta_2,\delta_3$ (corresponding to the $y$'s). They satisfy the relations $\bar\e_i+\bar\delta_i=0$ and $\sum \bar\e_i=0$.
 
 Recall that in the biggest quotient, I have 6 pieces that are being clued together. The two other quotients we got were obtained by taking just the $\bar\e_i$ and we got the other quotient by taking just the $\bar\delta_i$. We should also be able to remove $\bar \e_3$ and $\bar\delta_3$ to get $\PP^1\times \PP^1$ as a quotient. To do that, we'd have to choose a character which would give something like $J_1=\{\e_3,\delta_3,\delta_1,\e_2\}$.
\end{example}
Geometric meaning of $\Sigma$. I can identify $M^*$ with the dual lattice to $M$, so $M^*$ is the lattice of 1-parameter subgroups of $Q$. If you think about $M^\perp\subseteq (\ZZ^n)^*$, you get the lattice of 1-parameter subgroups in $T$.

There is an open $Q$-orbit $Y^0\subseteq Y$ which is isomorphic to $Q$ itself if none of the $\sigma$ contains a line (such a line would correspond to a subtorus that stabilizes every point). \anton{how can such a bad cone ever happen?}

Given a generic point $y\in Y^0$, we can define $M^*_0=\{\lambda\in M^*|\lim_{t\to 0}\lambda(t)y$ exists$\}$. We say that $\lambda\sim\mu$ if $\lim_{t\to 0} \lambda(t)y=\lim_{t\to 0}\mu(t)y$. Note that this is independent of choice of $y\in Y^0$. Each equivalence class is the interior of a face. To see this, identify $k[M]$ with $k[y_1^{\pm1},\dots, y_\ell^{\pm 1}]\supset R_\sigma$. Pick a monomial $y^b$.
\[
 \lim_{t\to 0} \lambda(t)y^b = 
 \begin{cases}
  y^b & \text{if }\<b,\lambda\>=0\\
  0 & \text{if }\<b,\lambda\>>0\\
  DNE & \text{if }\<b,\lambda\><0\\
 \end{cases}
\]
So we're checking locally in each open affine which coordinates become zero.

If $Y$ is complete, then $M_0^*=M^*$ because limits must always exist. In other words, $\bigcup \sigma_i=M^*_\RR$.

\begin{corollary}
 The faces are in bijection with the orbits of $Q$ in $Y$. Moreover, the dimension of a face is equal to the dimension of the stabilizer of a point in the orbit. So if the fan is non-degenerate, the origin corresponds to the open orbit.
\end{corollary}
$\sigma$ is simplicial if it is spanned by linearly independent rays (i.e. it's generated by a partial basis of $\RR^n$). A fan is simplicial if every cone in it is simplicial.
\begin{proposition}
 Suppose that $\ker\chi_1\cap\cdots\cap \ker\chi_n$ is finite. $\Sigma$ is simplicial if and only if $\AA^n(L_\chi)^{ss}=\AA^n(L_\chi)^s$.
\end{proposition}
\begin{proof}
 Suppose $\sigma$ is not simplicial, then we have some relation $\sum_{i\not\in J_\sigma} c_i\bar\e_i=0$. Then $\sum c_i\e_i\in M^\perp$. This $\sum c_i \e_i$ corresponds to a 1-parameter subgroup $\lambda(t)$ of $T$. Choose $x\in \AA^n$ such that $x_i=1$ for $i\in J_\sigma$ and $x_i=0$ for $i\not\in J_\sigma$. Then $\lambda(t)\in Stab(x)$, so $x$ is semi-stable but not stable.
\end{proof}
