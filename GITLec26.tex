\sektion{26}{Chow quotients}

Next the plan is to talk about Chow quotients. Then we'll need moment maps. That will probably take at least a week.

Let $X$ be a projective variety with an action of a reductive group $G$. We defined the Chow quotient $X\quot_C G$. There exists a Zariski open set $U\subseteq X$ such that for any two points $x,y\in U$, the orbit closures $\bbar{Gx}$ and $\bbar{Gy}$ represent the same homology class as algebraic cycles. Such orbits are called generic orbits. You can look on an affine set, then look at invariant functions $f_i(x)=c_i$ cutting out the orbit. If you perturb them a bit, then you'll get the same homology class. You can take $U$ as small as you like, so you can just take the regular orbits. Then you get a geometric quotient $U/G$. \anton{$U$ may not be $G$-invariant. you just restrict the equivalence relation generated by $G$ to $U$, and $U/G$ is the quotient by this relation}

We defined the Chow variety $C_r(X,\delta)$, the projective variety parameterizing all algebraic cycles of dimension $r$ which represent homology class $\delta$. Later we'll give a construction of this variety.

We get an embedding $U/G\hookrightarrow C_r(X,\delta)$. The Chow quotient is defined as the closure of the image. \anton{it's probably obvious that the map is quasi-compact and quasi-separated so you get a scheme-theoretic closed image}

\begin{example}
 Consider $\AA^4\subseteq \PP^4$ with the action of $k^\times$ given by $t(x_1,x_2,x_3,x_4)=(tx_1,tx_2,t^{-1}x_3,t^{-1}x_4)$. We discussed before that we have three quotients $Y_0$, $Y_+$, and $Y_-$, where $Y_0$ is the usual affine quotient, given by $z_1z_2=z_3z_4$. We had two maps $Y_+\to Y_0$ and $Y_-\to Y_0$, each of which is kind of a ``partial blowup'' of the cone point. This is related to the fact that we have the decomposition $k^4=V_+\oplus V_-$. Depending on the linearization, we get that the semi-stable points are the complement of $V_+$ or $V_-$.
 
 What is the Chow quotient in this case? Every orbit $G\cdot x$ can be projected onto $V_+$ or $V_-$. It's not hard to see that under each projection, the image is a line. So associated to each orbit are two lines $\ell_1$ and $\ell_2$, with $G\cdot x\subseteq \ell_1\oplus \ell_2$. It's actually a hyperbola, given by $uv=c$, where $u$ and $v$ are coordinates on $\ell_1$ and $\ell_2$. As soon as $c\neq 0$, all these orbits are rationally equivalent, so we take our $U$ to be the union of all of those. If $c=0$, then the hyperbola degenerates and you get a ``limit algebraic cycle''. This cycle $\ell_1+\ell_2$ is actually the union of three orbits, $\ell_i\setminus\{0\}$ and $\{0\}$. Notice that $\ell_1\setminus\{0\}$ is a semi-stable orbit in one case, and $\ell_2\setminus\{0\}$ is semi-stable in the other case. Note that the degenerate cross is rationally equivalent to the hyperbola, so the cross is the limit in the Chow variety. The Chow variety is very large. The Chow quotient has a point for each hyperbola and an extra point for the degenerate cross for each pair of lines. Q: you don't have a canonical choice of coordinates on the $\ell_i$. A: yes, but it doesn't matter.
 
 So the (affine part of the) Chow quotient $Y$ has maps to $Y_+$ and to $Y_-$. \anton{why?} We also have a map $Y\to Y_0$. There is a general theory that these affine maps can be extended. The preimage of the singular point in $Y_0$ is $\PP^1\times \PP^1$. This is in fact the blowup of the singular point of $Y_0$. It's a good exercise to check what's going on at infinity, but I won't do it now.
 
 There is a paper by Sturmfels-Kapranov-somebody else where they describe Chow quotients for toric varieties in general.
\end{example}
\begin{remark}
 Note that the Chow quotient \emph{is not} a categorical quotient. That is, there is no regular morphism $\bbar X\to \bbar Y$. In the example, the orbit $\ell_0\setminus\{0\}$ appears in many limit cycles, so there is no natural way to send it to a single limit cycle. You should also be able to see it from the toric description.
\end{remark}
The following is from a paper \anton{from Kapranov, Chow quotient of Grassmanians, I.M. Gelfand seminar \url{http://arxiv.org/abs/alg-geom/9210002}, Theorem 0.3.1. There is also a paper of Hu}
\begin{proposition}
 Let $X$ be a smooth projective variety and $G$ a reductive group acting on $X$. Assume the stabilizer $G_x$ is finite for generic $x\in X$. Moreover, assume that $G_x$ is not unipotent for any $x\in X$. Let $Z=\sum n_iZ_i$ be a cycle in the chow quotient (i.e.~a point on the Chow variety). Then every irreducible component $Z_i$ is the closure of exactly one orbit.
\end{proposition}
In the example, we have the cross, which has two components, each of which is the closure of an orbit.
\begin{proof}
 \anton{proof in the case of a torus} Let $G$ be a torus. Suppose the statement is false, so there is some $Z_i$ which is not the closure of a single orbit. Then there is a rational invariant function $f$ on $Z_i$ which is not constant \anton{linearize the action, look at lattice of laurent polynomials, so there must be an invariant laurent polynomial because the dimension of the variety is greater than that of the torus}\anton{For a torus, you can always find an affine open neighborhood of any orbit. Throw away some orbits to get a non-constant regular function. You can always throw away something so that the action is closed. Cover by affines, and separate orbits by invariants. Invariants on open sets give you rational functions on the whole space.}\anton{the invariant rational function on Z can be lifted because it is a ratio of semi-invariants. Semi-invariants can be lifted by complete reducibility; the same reason you can lift invariant regular functions}. But since $G$ is a torus, the rational function is a ratio of two semi-invariants \anton{was one of the exercises}, $f=p(x)/q(x)$, and every semi-invariant is a regular semi-invariants on $Z_i$. So we can lift them to get $p(x),q(x)\in k[V]^G$ (where $X\subseteq \PP(V)$). $Z=\lim_{t\to 0}Z(t)$, where $Z(t)$ is the closure of one orbit, so $f$ is constant on $Z(t)$ for any $t\neq 0$, so by continuity, it must be constant on $Z$, a contradiction.
\end{proof}
\begin{remark}
 In general, the proposition is not true. Consider the diagonal action of $SL(2)$ on $\CC^2\times \CC^2=\{(v_1,v_2)|v_i\in \CC^2\}$. We can naturally embed into $\PP^2\times \PP^2$ and extend the action in the natural way.
 
 What are the generic orbits? $SL(2)$ preserves determinant and can map one pair to another pair, so we get an invariant $\det(v_1,v_2)$. If $v_1$ is not proportional to $v_2$, we get a generic cycle in the Chow quotient, and it degenerates to the case $\det(v_1,v_2)=0$, in which case we get that the two are proportional, a 3-dimensional set. In this set, we have a 1-parameter family of orbits. $v_1=\lambda v_2$ is an orbit. Clearly the proposition is not true in this case because the cycle $\det(v_1,v_2)$ cannot be the closure of an orbit because the orbits are 2-dimesional.
 
 The point is that in this case, the stabilizer of $(e_1,\lambda e_1)$ is $\matx{1&*\\ 0&1}$, which is unipotent.
\end{remark}
How do we finish the proof of this proposition? There's a proof with Luna's slice theorem. Pick up a generic point $x\in Z_i$. If the proposition fails, then the dimension of the stabilizer of $x$ is positive by dimension reasons, so there is a non-trivial torus $T$ in the stabilizer since the stabilizer is not unipotent. For the torus, the proposition is already proven. Pick a curve $x(t)$ such that $x(0)=x$ and $x(t)$ is generic (in the sense of Chow quotient) if $t\neq 0$. Now we can tak the Chow quotient with respect to $T$. Consider the cycle $C=\sum m_i C_i$ which is the limit of $\bbar {T\cdot x(t)}$ as $t\to 0$. This cycle is contained in $Z$. So we can find $y_i\in C_i$ such that $x\in \bbar{Ty}$. From this, we get that $x\in \bbar{Gy}$, but we have to check that $x\not\in Gy$ \anton{this is the part I don't have an argument for}. Then we are done by dimension considerations. 