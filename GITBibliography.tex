%%%% The following three lines are used to produce a bibliography, then the contents of the bbl file are pasted below %%%%
% \nocite{mukai,git,stanley2,ega}
% \bibliographystyle{alpha}
% \bibliography{GITBib}

\newpage
\begin{thebibliography}{MFK94}

\bibitem[EGA]{ega}
Alexander Grothendieck and Jean-Alexandre-Eug\`ene Dieudonn\'e.
\newblock {\em \'{E}l\'ements de g\'eom\'etrie alg\'ebrique.}
\newblock Number 4, 8, 11, 17, 20, 24, 28, and 32. 1960-1967.
\newblock Institut des Hautes \'Etudes Scientifiques. Publications
  Math\'ematiques.

\bibitem[MFK94]{git}
D.~Mumford, J.~Fogarty, and F.~Kirwan.
\newblock {\em Geometric invariant theory}, volume~34 of {\em Ergebnisse der
  Mathematik und ihrer Grenzgebiete (2) [Results in Mathematics and Related
  Areas (2)]}.
\newblock Springer-Verlag, Berlin, third edition, 1994.

\bibitem[Muk03]{mukai}
Shigeru Mukai.
\newblock {\em An introduction to invariants and moduli}, volume~81 of {\em
  Cambridge Studies in Advanced Mathematics}.
\newblock Cambridge University Press, Cambridge, 2003.
\newblock Translated from the 1998 and 2000 Japanese editions by W. M. Oxbury.

\bibitem[Sta99]{stanley2}
Richard~P. Stanley.
\newblock {\em Enumerative combinatorics. {V}ol. 2}, volume~62 of {\em
  Cambridge Studies in Advanced Mathematics}.
\newblock Cambridge University Press, Cambridge, 1999.
\newblock With a foreword by Gian-Carlo Rota and appendix 1 by Sergey Fomin.

\end{thebibliography}
