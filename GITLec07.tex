\sektion{7}{Examples of Quotients by Finite Groups}

\begin{example}
 $S_n$ acts on $\AA^n$ by permuting the coordinates: $\pi(x_1,\dots, x_n)=(x_{\pi(1)},\dots, x_{\pi(n)})$. In this case, the invariants $R^G$ is the algebra of symmetric functions $k[\sigma_1,\dots, \sigma_n]$. $S_n$ is generated by reflections (the transpositions). For the reflection $s=(i\ j)$, we get $\ell_s=x_i-x_j$. We can verify that $d_1d_2\cdots d_n=n!=|S_n|$. We also know that $\sum (d_i-1)=n(n-1)/2$ is the number of reflections.
 
 Now let's look for semi-invariants. $S_G$ has just one orbit $O$, for which $f_O=\prod_{i<j}(x_i-x_j)=\Delta$, the Vandermonde determinant. We have that $\pi(\Delta)=\mathrm{sgn}(\pi)\Delta$.
\end{example}
\begin{example}
 Now consider the subgroup $A_n\subseteq S_n$. Since every element of $R^{A_n}$ is invariant under $A_n$, the action of $S_n$ is really an action of $\ZZ/2$. So $R^{A_n}$ decomposes (as a vector space) into a subspace invariant under the action of $S_n$ and a subspace where $S_n$ acts by $-1$. \anton{more generally, if $H\triangleleft G$ with $G/H$ abelian, then the action of $G$ on $R^H$ is really the action of the abelian group $G/H$, so it decomposes into isotypic components for irreps of an abelian group. Since irreps of an abelian group are 1-dimensional, $R^H$ has a basis of semi-invariants}. We have that $k[x_1,\dots, x_n]^{A_n}=k[\sigma_1,\dots, \sigma_n]\oplus \Delta k[\sigma_1,\dots, \sigma_n]$. We have that $\Delta^2$ is some specific symmetric polynomial that you can express in terms of symmetric functions.
\end{example}
\begin{example}
 Consider the group $\Ga$ of symmetries of a cube (not just rotations). We have that $|\Ga|=48$. It is generated by reflections, but it should be clear that there are two types (conjugacy classes) of reflections. If the coordinates of the cube are $(\pm 1,\pm 1,\pm 1)$, then one type of reflection is with respect to $x_i^\perp$, and the other with respect to $(x_i\pm x_j)^\perp$.
 
 First let's find invariants and semi-invariants. We know that $d_1+d_2+d_3=12$ and $d_1d_2d_3=48$. There is one solution: $(d_1,d_2,d_3)=(2,4,6)$. We have one invariant given by the quadratic form: $h_2=x_1^2+x_2^2+x_3^2$. But we also have $h_4=x_1^4+x_2^4+x_3^4$ and $h_6=x_1^6+x_2^6+x_3^6$. These are the invariants. We have semi-invariants $\tilde h_3=x_1x_2x_3$, $\tilde h_6=\prod_{i<j}(x_i\pm x_j)$.
\end{example}
\begin{example}
 Now let's take $H\subseteq \Ga$ the subgroup of rotations of the cube. Clearly the invariants $h_i$ remain invariants. But we get one more invariant: $h_9=\tilde h_3\tilde h_6$. You know you have all the invariants because the subgroups is index 2 \anton{so the ring of invariants is a module of rank 2 over the other guy?}.
\end{example}
\begin{example}
 Recall that we have a 2-fold cover $\ga\colon SU(2)\to SO(3,\RR)$. We got this by considering the squaring map $\CC^2\to \sym^2\CC^2\cong \CC^3$. Let $G=\ga^{-1}(H)$ (with $H$ as before). We get three invariants: $f_8$, $f_{12}$, and $f_{18}$.
 
 Since $\ga$ is of degree 2, all degrees multipy by 2. $f_4=\ga^{-1}(h_2)$, $f_8=\ga^{-1}(h_4)$, $f_{12}=\ga^{-1}(h_6)$, and $f_{18}=\ga^{-1}(\tilde h_9)$. But since $\ga(V)$ is the vanishing locus of $h_2$, that invariant goes away. We get the relation
 \[
  f_{18}^2+f_{12}f_8^3+f_{12}^3=0
 \]
 So the corresponding quotient is the surface $x^2+yz^3+y^3=0$. This is what is called a \emph{simple singularity}.
\end{example}
\underline{Simple singularities}. Consider locally maps like $\phi\colon \CC^m\to \CC$ with $\phi(0)=0$. We'd like to classify germs of such maps at $0$ \anton{something to do with classifying orbits}. If non-singular, then easy, but if singular, then we have to work harder.

Smooth map $\phi$ is \emph{simple} (or $\phi^{-1}(0)$ is a simple singularity) if whenever we deform it a little bit, there are only finitely many orbits (under the action of the group) in a small neighborhood. That is, a small neighborhood of $\phi$ intersects non-trivially with a finite number of orbits of $G$. \anton{orbits of the group $\ttilde{\mathrm{Diff}}(\CC^m)\times \ttilde{\mathrm{Diff}}(\CC)$ (the twiddle means you work germy).}

\subsektion{Finite subgroups of \texorpdfstring{$SU(2)$}{SU(2)}}

Given $G\subseteq SU(2)$, we get an image group in $SO(3)$. Let's classify these groups. The main result is that you get rotation groups, dihedral groups, and rotations of polytopes.

$G\subseteq SO(3,\RR)$ acts on $S^2$. Consider $P=\{p\in S^2|\mathrm{Stab}(p)\neq \{1\}\}$. This is a finite set with a $G$-action, and they will correspond to vertices, edges, and faces of the polytope.

Write $P$ as a union of orbits $P=P_1\sqcup \cdots \sqcup P_d$. Let $g=|G|$, and let $g_i=|\mathrm{Stab}(p_i)|$ for any $p_i\in P_i$. Then $\{(g,p)|p\in P,g\in G\setminus \{1\}, gp=p\}$ can be counted in two ways. Counting elements of the group: each non-trivial rotation fixes two points, so we get $2\cdot (g-1)$. On the other hand, we can count by orbits, in which case we get $\sum_{i=1}^d \frac{g}{g_i}(g_i-1)$. So we get
\[
 2g-2 = gd-\sum \frac{g}{g_i} \qquad\Rightarrow\qquad d=2+\sum_{i=1}^d \frac{1}{g_i}-\frac 2g.
\]
From this, we can see that $d$ can only take values 1, 2, or 3. The $g_i$ are never 1, so the sum is less than or equal to $\frac 12 |G|$. You can check that $d=1$ is impossible. For $d=2$, there is only one possibility: $g_1=g_2=g$, which happens when the group is generated by a single rotation, so it is a cyclic group. Finally, if $d=3$, we can write $1+\frac 2g = \sum \frac 1{g_i}$, so the $g_i$ cannot be too big. One of them must be 2, so the possibilities are
\begin{itemize}
 \item $d=3$, $g_1=g_2=2$, $g_3=g/2$. You get the north pole, south pole, and a polygon on the equator. This is the dihedral group.
 \item $d=3$, $g_1=2$, $g_2=g_3=3$, $g=12$. Symmetries of a tetrahedron.
 \item $d=3$, $\{g_i\}=\{2,3,4\}$, $g=24$. Rotations of the cube, $S_4$.
 \item $d=3$, $\{g_i\}=\{2,3,5\}$, $g=60$. Rotations of the dodecahedron, $A_5$.
\end{itemize}

We're interested in subgroups of $SU(2)$. In the first case ($d=2$) you get the same group, but in all the other cases, you get a non-trivial central extension. For the dihedral case, you get $\< \matx{\e&0\\ 0&\e^{-1}}, \matx{0&1\\ -1&0}\>$

\begin{tabular}{l|lll}
 group & invariants & relations & type \\ \hline
 \rule{0pt}{1em}$\ZZ/n$ & $f_2=xy$, $f_n=x^n$, $\bar f_n=y^n$ & $f_n\bar f_n-f_2^n=0$ & $A_{n-1}$\\
 $\tilde D_n$ & $f_{2n},f_{2n+2}, f_4$ & $f_{2n+2}^2+f_4f_{2n}^2+f_4^{n+1}$ & $D_{n+1}$ \\
 tetrahedron& $f_6, f_8, f_{12}$ & $f_6^4+f_8^3+f_{12}^2$ & $E_6$\\
 cube & $f_8,f_{12},f_{18}$ & $f_{18}^2+f_{12}f_8^3+f_{12}^3$ & $E_7$\\
 dodeca & $f_{12},f_{20},f_{30}$ & $f_{30}^2+f_{20}^3+f_{15}^5$ & $E_8$
\end{tabular}

For a simple singularity $\phi(x,y,z)=0$, call it $X_0$. We deform to get $\phi(x,y,z)=\e$, which we call $X_\e$. Study it in a neighborhood of $0$. $H_2(X,\ZZ)$ is a lattice, equiped with a quadratic form (index of intersection). On the other hand, you have a Dynkin diagram, which gives you the root lattice, which will be isomorphic (with it's usual form) to $H_2(X,\ZZ)$.

We've been talking about maps $\CC^3\to \CC$. But there is a theorem that for any $\CC^m\to \CC$, you get the same singularities. Any singularity is of the form $(x^2+xy+z^3) + (x_1^2+\cdots x_m^2)$, a surface singularity plus some quadratic part.

In the theory of Lie groups and the theory of algebraic groups, there is a very small difference between the usual Dynkin correspondence. I'd like to point out why the characteristic 0 case is very nice, and why we get many more reductive groups in characteristic 0.

\subsektion{Lie algebras and algebraic groups}

Assume that $X$ is an affine variety. We have a tangent space $T_x X=(\m_x/\m_x^2)^*$, where $\m_x$ is the maximal ideal in the local ring at $x$. We can think of it as $T_xX=Der\{\O_x\to k\}$. Or we can think of vector fields $Der(\O(U),\O(U))$ for any open set $U\subseteq X$. (If you don't work with an affine variety, you need to take sheafy derivations). In our case, $X=G$ is an affine non-singular variety (because the dimension of the tangent space is uniform).

For derivations $d_1,d_2$, the bracket $[d_1,d_2]=d_1\circ d_2-d_2\circ d_1$ is again a derivation. We can consider \emph{right-invariant} derivations. Those derivations which commute with the maps $R_g\colon G\to G$, given by multiplication by $g$ on the right. $\g=\{d|d\circ R_g=R_g\circ d\}$. In terms of hopf algebras, $\Delta\circ d=(d\otimes \id)\circ \Delta$.

