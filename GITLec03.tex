\sektion{3}{Reductive Groups acting on Affine Varieties}

\subsektion{The Reynolds operator and Hilbert's Theorem}

Suppose $G\times X\to X$ is an action of a reductive group $G$ on an affine variety $X$. There is a corresponding ring morphism $\sigma_X\colon k[X]\to k[G]\otimes k[X]$, and the axioms of a group action exactly correspond to this dual morphism being a coaction, so $R=k[X]$ is a representation of $G$. The fact that this coaction is a ring homomorphism (rather than just a $k$-linear map) corresponds to the fact that $G$ acts on $R$ is by ring automorphisms (rather than just $k$-linearly).

If $X$ is a variety of finite type, then $R$ is finitely generated algebra over the ground field: $R=k[x_1,\dots,x_n]/I$. Since $G$ is reductive, then we get the decomposition $R=R^G\oplus R'$, so in addition to the obvious embedding $R^G\hookrightarrow R$, we have a canonical projection $R\to R^G$,\footnote{The projection is canonical because the invariant complement $R'$ is canonical. It is the direct sum of all the non-trival isotypic components of $R$.} called the \emph{Reynolds operator}, which we will denote $f\mapsto \bar f$.\anton{should mention Reynolds operator earlier \dots the term is used for the projection $V\to V^G$ in general}
\begin{warning}
 This projection is not a homomorphism of rings since $R'$ is not an ideal of $R$.\footnote{For example, consider the action of $G=\mu_2$ on $R=k[x]$ given by $x\mapsto -x$. Then $R^G=k[x^2]$ and $R'=xk[x^2]$.} However, the following lemma shows that $R'$ is an $R^G$-module, so the projection is a homomorphism of $R^G$-modules.
\end{warning}
\begin{lemma}\label{lec3Lem:R^G-linear}
 $R'$ is an $R^G$-module. In particular, the projection $R\to R^G$ is $R^G$-linear.
\end{lemma}
\begin{proof}
 By definition of $R^G$, we have that $\sigma_X(f)=1\otimes f$. Since $\sigma_X$ is a ring homomorphism, we have $\sigma(fh)=(1\otimes f)\cdot \sigma(h)$ for any $h\in R$. That is, multiplication by $f$ is a morphism of representations $R\to R$. By Remark \ref{lec2Rmk:isotypic}, it must respect isotypic components. In particular, it must send $R^G$ to $R^G$ and $R'$ to $R'$.
\end{proof}
\begin{theorem}[Hilbert]\label{lec3Thm:Hilbert}
 Suppose $R=k[x_1,\dots, x_n]=\bigoplus_{d\ge 0} R_d$ (the natural grading, where each $x_i$ has degree 1), and $G$ is a reductive group acting on $\spec R$ such that the action of $G$ respects the grading, so $G\cdot R_d=R_d$ (or $\sigma(R_d)\subseteq k[G]\otimes R_d$). Then $R^G$ is a finitely generated (graded) $k$-algebra.
\end{theorem}
\begin{proof}
 It is clear that $R^G$ inherits a grading from $R$. Consider $R^G_{>0} = \bigoplus_{d>0}R^G_d=\bigl(\bigoplus_{d>0}R_d)^G$. Let $I$ be the graded ideal in $R$ generated by $R_{>0}^G$. Since $R$ is noetherian, there is a finite set of homogeneous $f_1,\dots, f_n\in R_{>0}^G$ which generate $I$. We will show that $f_1,\dots, f_n$ generate $R^G$ as a ring.
   
 Given $h\in R^G_d$, we want to show that $h\in k[f_1,\dots, f_n]$. Since $h\in I$, we have $h=\sum f_i r_i$ for some homoegenous $r_i\in R$. Applying the projection to $R^G$, and using that $h,f_i\in R^G$ and Lemma \ref{lec3Lem:R^G-linear}, we have $h=\sum f_i\bar r_i$. Now each $\bar r_i$ is a homogeneous element of $R^G$ of degree less than $d$, so by induction on $d$, $\bar r_i\in k[f_1,\dots, f_n]$, so $h\in k[f_1,\dots, f_n]$ as desired.
\end{proof}
\begin{lemma}\label{lec3Lem:closed_immersion}
 Suppose an algebraic group $G$ (which need not be reductive) acts on an affine variety $X$. Then there exists a $G$-equivariant closed embedding $X\hookrightarrow \AA^n$ such that the action of $G$ on $X$ extends to a \emph{linear} action of $G$ on $\AA^n$.
\end{lemma}
\begin{proof}
 Let $s_1,\dots, s_n$ be generators for $k[X]$ such that the subspace generated by the $s_i$ is an invariant subspace (we proved that each generator lies in a finite-dimensional invariant subspace, so take the sum of all of those). If $\sigma_X(s_i)=\sum_j f_{ij}\otimes s_j$, define an action of $G$ on $k[x_1\dots, x_n]$ by $\sigma(x_i)=\sum_j f_{ij}\otimes x_j$. This is a linear action of $G$ on $\AA^n$, and we have an invariant map $k[x_1,\dots, x_n]\to k[X]$ given by $x_i\mapsto s_i$. This is a surjection since the $s_i$ were chosen to generate $k[X]$, so it induces a closed immersion.
\end{proof}
\begin{remark}
 Note that a linear action of $G$ on $\AA^n$ is the same thing as a $G$-comodule structure on $k[x_1,\dots, x_n]$ that respects the grading.
\end{remark}
\begin{corollary}[Hilbert's Theorem + Lemma \ref{lec3Lem:closed_immersion}]\label{lec3Cor:Hilbert}
 If a reductive group $G$ acts on an affine variety $X$, with coordinate ring $R=k[X]$, then the ring of invariants $R^G$ is finitely generated.
\end{corollary}
\begin{proof}
 By Lemma \ref{lec3Lem:closed_immersion}, we get a surjection of comodules $k[x_1,\dots, x_n]\twoheadrightarrow R$, where the action of $G$ on $k[x_1,\dots, x_n]$ respects the grading. Since $G$ is reductive, we get a surjection $k[x_1,\dots, x_n]^G\to R^G$. By Hilbert's Theorem (\ref{lec3Thm:Hilbert}), $k[x_1,\dots, x_n]^G$ is finitely generated, so $R^G$ must also be finitely generated.
\end{proof}

\subsektion{The Orbit-closure Relation and Separation Lemma}

For an arbitrary algebraic group $G$ acting on a variety $X$, the inclusion $R^G\hookrightarrow R$ induces a map on spectra $\phi\colon X=\specm R\to \specm R^G=:X\quot G$ whose properties we'd like to study.
\begin{remark}\label{lec3Rmk:invariant_ideals}
 If $G$ is a reductive group and $I\subseteq R$ is an invariant ideal, then $I$ is a subrepresentation of $R=R^G\oplus R'$, so $I=(I\cap R^G)\oplus (I\cap R')$ (c.f.~Remark \ref{lec2Rmk:isotypic}).
\end{remark}
\begin{lemma}\label{lec3Lem:(Rn)^G=n}
 If $G$ is a reductive group and $\mathfrak n\subseteq R^G$ is an ideal, then $(R \mathfrak n)^G=\mathfrak n$. In particular, if $\mathfrak n$ is proper, then $R \mathfrak n$ is proper.
\end{lemma}
\begin{proof}
 We have that $R \mathfrak n$ is an invariant ideal, so by Remark \ref{lec3Rmk:invariant_ideals}, $R\mathfrak n=(R\mathfrak n\cap R^G)\oplus (R\mathfrak n\cap R')$. So $(R\mathfrak n)^G= R\mathfrak n\cap R^G = \mathfrak n + (R'\mathfrak n\cap R^G)$. By Lemma \ref{lec3Lem:R^G-linear}, $R'\mathfrak n\subseteq R'$, so $(R\mathfrak n)^G=\mathfrak n$.
\end{proof}
\begin{corollary}\label{lec3Cor:phi_surjective}
 If $G$ is reductive, $\phi\colon X\to X\quot G$ is surjective.
\end{corollary}
\begin{proof}
 For any maximal ideal $\m\in \specm R^G$, $R\m\cap R^G=\m$ by Lemma \ref{lec3Lem:(Rn)^G=n}. So $R\m$ is contained in some proper maximal ideal $\mathfrak M\in \specm R$. We have that $\mathfrak M \cap R^G=\m$ (since $\m$ is maximal in $R^G$ and $\mathfrak M\cap R^G$ cannot contain $1$), so $\phi(\mathfrak M)=\m$.
\end{proof}
\begin{lemma}[Separation Lemma]\label{lec3Lem:separation}
 Suppose $Z_1$ and $Z_2$ are Zariski-closed $G$-invariant subsets of $X$ with $Z_1\cap Z_2=\varnothing$. Then there is an invariant function $f\in R^G$ such that $f(Z_1)=0$ and $f(Z_2)=1$.
\end{lemma}
\begin{proof}
 Let $I_1$ and $I_2$ be the ideals of $Z_1$ and $Z_2$. Since $Z_1\cap Z_2=\varnothing$, $I_1+I_2=R$, so we may write $1=g_1+g_2$ for $g_i\in I_i$. Applying the projection to $R^G$, we have $1=\bar g_1+\bar g_2$. Since $I_i$ are invariant ideals, $\bar g_i\in I_i$ by Remark \ref{lec3Rmk:invariant_ideals}. Take $f=\bar g_1$.
\end{proof}
\begin{definition}
 If $G\times X\to X$ is an action of an algebraic group on a variety and $x\in X$ is a point, then the \emph{orbit} $G\cdot x$ of $x$ is the image of the restricted map $G\times \{x\}\to X$. \anton{I'd like to add a functorial definition, but it's probably not worth it.}
\end{definition}
Note that any invariant function must be constant along an orbit, so each fiber of $\phi$ is a union of orbits.
\begin{lemma}\label{lec3Lem:open_in_closure}
  Every orbit $G x$ is open in its Zariski closure.
\end{lemma}
\begin{proof}
 By Chevalley's constructibility theorem \cite[Theorem IV$_\text{1}$.1.8.4 and Proposition 0$_\text{III}$.9.2.2]{ega}, for any finitely presented morphism of varieties $f\colon A\to B$, $f(A)$ contains an open set of $\bbar{f(A)}$. So $Gx$ contains an open subset $U$ of $\bbar{Gx}$. But then $Gx\bigcup g(U)$, which is open.
\end{proof}
\begin{remark}
 In the differential category, this lemma can be false. For example, you can wrap a 1-parameter subgroup around a 2-dimensional torus so that the subgroup is dense. In the algebraic category, such nasty things can't happen.
\end{remark}
\begin{definition}
 Two orbits $O$ and $O'$ are \emph{closure equivalent} if there exists a finite set of orbits $O=O_0,O_1\dots, O_n=O'$ such that $\bbar O_i\cap \bbar O_{i+1}\neq \varnothing$.
\end{definition}
If $O\sim O'$, then it is clear that all invariants agree on them, so $\phi(O)=\phi(O')$.
\begin{proposition}\label{lec3Prop:orbit_closure}
 Suppose a reductive group $G$ acts on an affine variety $X$, and that $O_1$ and $O_2$ are two orbits, then the following conditions are equivalent.
 \begin{enumerate}
  \item $O_1\sim O_2$
  \item $\phi(O_1)=\phi(O_2)$
  \item $\bbar O_1\cap \bbar O_2\neq\varnothing$
 \end{enumerate}
\end{proposition}
\begin{proof}
 We've already proven $(1\Rightarrow 2)$ since all invariants agree on these orbits.
 
 $(2\Rightarrow 3)$ Suppose $\bbar O_1\cap \bbar O_2=\varnothing$, then by the Separation Lemma (\ref{lec3Lem:separation}), there is an inivariant which is $1$ on one of them and $0$ on the other, contradicting $(1)$.
 
 $(3\Rightarrow 1)$ trivial.
\end{proof}
\begin{corollary}
 So the fibers of $\phi\colon \specm R\to \specm R\quot G$ are closure equivalence classes of orbits.
\end{corollary}
\begin{example}
 $G=k^\times$ acts on $\AA^2$ by $t\cdot (x,y)=(tx,t^{-1}y)$. It is not difficut to see that $R^G=k[xy]$. The fibers are $\{xy=C\}$ If $C\neq 0$, there is just one orbit in the fiber (a hyperbola in the real picture). But if $C=0$, then the fiber consists of three orbits, $\{x\neq 0\}$, $\{y\neq 0\}$, and $\{(0,0)\}$. But note that the fiber contains only one closed orbit.
\end{example}
\begin{proposition}
 For a reductive group acting on an affine scheme, every closure equivalence class has exactly one closed orbit.
\end{proposition}
\begin{proof}
 (Existence) Pick an orbit $O$ of minimal dimension. By Lemma \ref{lec3Lem:open_in_closure}, $O$ is open in $\bbar O$. But $\bbar O$ is invariant, so it is a union of orbits. If $\bbar O$ contained any orbit other than $O$, that orbit would have to be of smaller dimension. Thus, $O=\bbar O$.
 
 (Uniqueness) Suppose $O_1$ and $O_2$ are two closed orbits in the same orbit-closure equivalence class. Then by Proposition \ref{lec3Prop:orbit_closure}, $O_1\cap O_2\neq\varnothing$. But the only way two orbits can intersect is if they are equal.
\end{proof}
In particular, every point in the geometric quotient corresponds to a closed orbit. In Example \ref{lec1:A^n/G_m} ($k^\times$ acting on $k^n$ by scaling), there is only one closed orbit, so the quotient has only one point.

\begin{corollary}
 If $G$ is reductive with a closed action on an affine variety, then there is a bijection between $G$-orbits and points of the geometric quotient.
\end{corollary}

\begin{example}[A non-reductive counterexample]
 Suppose $G\cong \GG_a=\CC$ acts on $\AA^2=\CC^2$ by $t\cdot (x,y)=(x+ty,y)$. In this case, $R^G=k[y]$. The horizontal lines $\{y=c\}_{c\neq0}$ are orbits; each one corresponds to a point in $X\quot G$. But every point on the line $\{y=0\}$ is fixed, so there is a line of closed orbits which get sent to the origin in $X\quot G=\specm k[y]$. This is a closed action where orbits do not correspond to points in the quotient because $\phi$ identifies some of the closed orbits.
\end{example}

\begin{example}
 $G=\ZZ/2$ acting on $\AA^2$ by $(x,y)\mapsto (-x,-y)$. Then $R^G=k[x^2,xy,y^2]=k[u,v,w]/(v^2-uw)$. Geometrically, $X\quot G$ is a quadratic cone in 3-dimensional space. \anton{does this example illustrate something special?}
\end{example}

\begin{proposition}
  Suppose a reductive group $G$ acts on an affine variety $X$, then $\phi\colon X\to X\quot G$ is an \emph{open submersion} (i.e.~the topology on $X\quot G$ is induced by $\phi$; $U\subseteq X\quot G$ is open if and only if $\phi^{-1}(U)\subseteq X$ is open).
\end{proposition}
\begin{proof}
 Since $\phi$ is surjective (Corollary \ref{lec3Cor:phi_surjective}), it suffices to show that for an invariant closed subset $Z\subseteq X$, the image $\phi(Z)\subseteq X\quot G$ is closed. Let $I\subseteq R$ be the $G$-invariant ideal corresponding to $Z$ (we're taking the reduced induced structure on $Z$). Then the ideal $I^G=I\cap R^G\subseteq R^G$ corresponds to the closure of the image of $Z$. So we need to show that the map $Z=\specm (R/I)\to \specm (R^G/I^G)=\bbar{\phi(Z)}$ is surjective. But since $G$ is reductive, invariants is exact, so $R^G/I^G\cong (R/I)^G$, so $\bbar{\phi(Z)}\cong Z\quot G$. Applying Corollary \ref{lec3Cor:phi_surjective}, we have that $Z\to \bbar{\phi(Z)}$ is surjective.
\end{proof}