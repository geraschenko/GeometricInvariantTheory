\sektion{31}{Convexity Theorems}

\begin{proposition}
 Let $M$ be a compact connected symplectic variety. Suppose $f\in C^\infty(M)$ such that $\om(df)=L_\zeta\in Vect(M)$ where $\zeta\in \t_f$ (the Lie algebra of some compact torus $T$ with a Hamiltonian action on $M$). Let $Z=\{x\in M|df(x)=0\}=Z_1\sqcup \cdots \sqcup Z_N$, where each $Z_i$ is a connected component of the cricical locus $Z$. Then
 \begin{enumerate}
  \item[(a)] Each $Z_i$ is a smooth manifold.
  \item[(b)] For $x\in Z_i$, the hessian $D^2f(x)|_{T_xM/T_x Z_i}$ is a non-degenerate quadratic form.
  \item[(c)] The form in (b) has even signature (even number of positive eigenvalues, so also an even number of negative eigenvalues, since $\dim M$ is even).
  \item[(d)] The restriction of $\om$ to $Z_i$ is non-degenerate, so $Z_i$ is symplectic.
 \end{enumerate}
\end{proposition}
\begin{remark}
 Say $K$ is compact (or reductive, in the algebraic case), and let $Z$ be the set of fixed points. Then $Z$ is non-singular. To see this, consider the action $\sigma\colon K\times M\to M$ and $d\sigma(x)\colon \k\oplus T_x M\to T_x M$. For a fixed point $x\in Z$, the group (and the Lie algebra) acts on the tangent space. The tangent space $T_x Z$ is just $(T_x M)^\k=(T_x M)^K$. This proof will work in the algebraic setting if the group is reductive. We have a decomposition $T_x M=(T_x M)^{K_0}\oplus N$, where $N$ is the normal bundle to $Z$. If $x$ is in a connected component of $Z$, $\dim (T_x M)^{K_0}$ is fixed because representations of a reductive or compact group are discrete, so the dimension of the normal bundle is fixed, so $Z$ is smooth. This proves part (a).
\end{remark}
\begin{proof}
 Let $\t_f=\lie T_f$. Then $Z$ is fixed by $T_f$ since the 1-parameter subgroup generated by $L_\zeta$ fixes $Z$.
 
 Let $x\in Z_i$, so $T_xM=T_xZ\oplus N$. We know that $\zeta$ infinitesimally preserves the form because the action is symplectic: $\om(\zeta v,w)+\om(v,\zeta w)=0$. We also know that the eigenvalues of $\zeta$ are purely imaginary because if we exponentiate, we get a unitary matrix. Consider $V_\CC=V\otimes_\RR \CC$. If $\zeta v=pv$ and $\zeta w=qw$ are eigenvectors, then $\om(v,w)\neq 0$ implies that $p+q= 0$. So we get a decomposition (after complexifying) $V_\CC = V_0\oplus V_1 \oplus \cdots \oplus V_k$ where $V_0=\ker \zeta$ and $\om(V_i,V_j)=0$ for $i\neq j$. So $\dim V_i=2$ and the restriction of $\zeta$ is $\matx{p_i&0\\ 0&-p_i}$. $T_x M=T_x Z\oplus V_1^\RR\oplus \cdots \oplus V_k^\RR$. Since $p_i$ was purely imaginary, the action of $\zeta$ on $V_i^\RR$ is $\matx{0& q_i\\ -q_i&0}$ where $q_i=\sqrt{-1}p_i$; we can choose a basis so that $\om(e_1,e_2)=1$. Locally, we can choose coordinates $x_1,\dots, x_k, y_1,\dots, y_k$ such that $H_\zeta = \sum_{i=1}^k q_i (x_i^2+y_i^2)+O(higher)$, so we know the terms that contribute to the signature of the hessian, and we see that the signature is even. Something about $q_i x_i\pder{}{y_i}-q_iy_i\pder{}{x_i}$. From the decomposition of $T_x M$, we see that the restriction of $\om$ to $T_x Z$ is non-degenerate since $\om(T_x Z,V_i)=0$.
\end{proof}
So we've proven that the preimage of a point under the moment map of a projection to a 1-dimensional space is connected.

\begin{proof}[Proof of Theorem]
 Assume the preimage of a point is connected for $n$ commuting Hamiltonians. Suppose we have $n+1$ commuting Hamiltonians $f_1,\dots, f_{n+1}$. We want to show $f_1^{-1}(c_1)\cap \cdots\cap f_{n+1}^{-1}(c_{n+1})$ is connected. We know that $N=f_1^{-1}(c_1)\cap \cdots\cap f_n^{-1}(c_n)$ is $T$-invariant and connected. We want to show that $f_{n+1}\colon N\to \RR$ satisfies (a), (b), and (c). Since the number of connected components of the fiber is semi-continuous, it is enough to prove the result for a generic $c_{n+1}$. So we may assume $df_1(x),\dots, df_n(x)$ are linearly independent for $x\in N$ (which we can assume because they generate the torus; if they aren't independent generically, we can remove one of them). Let $x\in N$ be a critical point of $f_{n+1}|_N$, so $df_{n+1}(x)+\sum_{i=1}^n b_i df_i(x)=0$. Consider $\phi=f_{n+1}+\sum b_i f_i$. Define $Z_\phi = \{y\in M| d\phi(y)=0\}$. Then by definition, $x\in N\cap Z_\phi$. By transversality, we have that $T_x(N\cap Z_\phi)=T_x N\cap T_xZ_\phi$. $\om(df_1)(x),\dots, \om(df_n)(x)\in T_xN$ because these are vector fields of the action of the group and $N$ is $T$-invariant. Since the $df_i$ are linearly independent, they are linearly independent when restricted to $T_xN$. All the arguments in the proof of the Proposition can now be repeated. $\phi$ is the hamiltonian of some vector field: $\om(d\phi)=L_\zeta$. This $\zeta$ has purely imaginary eigenvalues and xxxx, so the argument works the same way.

 Once you know that the indexes of critical points are even, you get connectivity of the fibers. This is some result from Morse theory (it's work of Bott?). If you have an isolated critical point, when you pass a critical point, it looks like $x_1^2+\cdots +x_k^2-x_{k+1}^2-\cdots -x_n^2=\e$, whose fibers are connected unless $k=1$ or $k=n-1$ (depending on the sign of $\e$).

 To finish the Theorem, we have the moment map $\mu\colon M\to \t^*=\RR^n$. We have (a) $\mu^{-1}(c)$ is connected. From this, we get (b) $\mu(M)$ is convex as we explained last time. Moreover, (c) if $Z=Z_1\sqcup \cdots Z_N=\{x\in M|d\mu(x)=0\}$, then $\mu(Z_i)=c_i$ (a single point!), and $\mu(M)$ is the convex hull of the $c_i$.
 
 Recall that $\mu=(f_1,\dots, f_n)$, where the $f_i$ are commuting Hamiltonians. Take an arbitrary linear combination $f=\sum b_i f_i$ for $b_i\in \RR$. If this combination is generic, then $T_f=T$ and $df=0$ if and only if $d\mu=0$. That means that the $Z_i$ go to one point since they are fixed points of $f$. And $\max_{m\in M}f$ is attained at one of the $Z_i$. So $\max_{x\in \mu(M)} (b_1x_1+\cdots b_nx_n)$ is attained at some $c_i$. So we have a convex set, and for a generic linear functional, the maximum is attained at one point, and there are a finite number of such points. This implies that the convex set is a polytope, the convex hull of those points.
\end{proof}
Let me talk a little bit about the second theorem. Suppose $M$ is a complex variety (think $\PP^n$) with K\"ahler structure. Let $T$ be a complex torus acting on $M$, preserving the K\"ahler structure. Again we get a moment map. For $y\in M$, consider $\overline Y=\overline{Ty}$, which is a toric variety. In most cases, this $\overline Y$ is singular, so we can't use the previous theorem. The moment map $\mu\colon \overline{Y}\to \t_K^*$ (the Lie algebra of a compact form of the torus). Then we get a bijection between faces of the image of $\mu$ and $T$-orbits in $\overline Y$. And the quotient $\overline Y/T_K$ is homeomorphic to the polytope $\mu(\overline Y)$.

The point is that a K\"ahler manifold has a symplectic structure and a Riemannian structure. We have two isomorphisms $\om,g:T^*M\to TM$ and $\om^*,g^*\colon TM\to T^*M$, with $\om\circ \om^*=-1$ (since it is skew-symmetric) and $g\circ g^*=1$ (since it is symmetric). The K\"ahler condition says that $J=g\circ \om^*=\om\circ g^*$ and it satisfies $J^2=-\id$.

We can consider $T$ as the product of $T_K$, the compact torus, and $H$. For $T=\CC^\times$, $H=\RR^+$ and $K=S^1$. Consider $\mu\colon M\to \RR^n$. Acting by $K$ doesn't change the image under $\mu$, but acting by an element of $H$ moves you around in the image of $\mu$. For some $h\in \lie H$ and $y\in \overline Y$, $\exp(hs)y$ is a curve in $M$ whose image we can look at under $\mu$. For generic $h$, the curve goes to a vertex. But if $h$ is perpendicular to a face, the curve flows to the face.