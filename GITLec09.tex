\sektion{9}{Classifying Reductive Groups, Part I}

Today we'll try to classify reductive groups in characteristic $p$.

Last time, we showed that any $g\in G$ has a natural decomposition $g=g_sg_u$. Let $G_s\subseteq G$ be the set of semi-simple elements and $G_u$ be the set of unipotent elements. In general $G_u\subseteq G$ is a Zariski closed set (the vanishing locus of $(1-x)^N$ for some $N$).

Fact from linear algebra: any commuting set of semi-simple (diagonalizable) matrices can be simultaneously diagonalized.

Consider the case where $G$ is an abelian group. In this case, $G_sG_s=G_s$ and $G_uG_u=G_u$, so we get a decomposition $G=G_s\times G_u$ as a group. We already know that $G_u$ is closed, and we get that $G_s$ is closed. This gives us a decomposition of the Lie algebra $\g=\g_n\oplus \g_u$.

\begin{proposition}
 Suppose $G=G_s$ is an abelian group. Then $G=\Ga\times G_0$, where $\Ga$ is a finite group (with $\mathrm{char}(k)\nmid \Ga$) and $G_0$ is isomorphic to a torus.
\end{proposition}
\begin{proof}
 We have $G\subseteq GL(V)$ for some $V$. Since all elements of $G$ are diagonalizable and they commute, they are simultaneously diagonalizable, so in some basis for $V$, $G$ is a closed subgroup of the group of diagonal matrices, so we are describing closed subgroups of a torus. So we have $G\subseteq T$ a closed subgroup in a torus. $R=k[T]\cong k[t_1^{\pm 1},\dots, t_n^{\pm 1}]$. Consider the character lattice $T^\vee$. Then $R=\bigoplus_{\chi\in T^\vee} k\cdot t^\chi$, where $t^\chi$ is the monomial such that $g\cdot t^\chi=\chi(g)t^\chi$. Define $L=\{\chi\in T^\vee| t^\chi(G)=1\}\subseteq T^\vee$.
 
 $T^\vee/L=\Ga^\vee\times \ZZ^m$ for some finite group $\Ga^\vee$. So $G=\Ga\times G_0$, where $G_0$ is a torus of rank $m$. \anton{We're using that $G^{\vee\vee}=G$ and that $-^\vee$ is exact here}
 
 \anton{next three paragraphs are to show that $p\nmid |\Ga|$} Consider $Q\subseteq R$, the linear subspace spanned by all expressions of the form $t^\chi-t^{\chi'}$ where $\chi,\chi'\in L$, and let $I^L$ be the ideal generated by $Q$. We have $I^L=\bigoplus_{\chi\in T^\vee/L}Q_\chi$, where $Q_\chi=t^\chi Q$. Note that $Q_1=Q$. It is clear that this ideal is invariant under the action of $G$ since $Q$ is invariant under the action of $G$. It is also clear that $I^L\subseteq I_G$ because it is generated by $Q$ and every element of $Q$ vanishes on $G$.
 
 Now $R=\bigoplus_{\chi\in T^\vee/L} R_\chi$, where $R_\chi=\{f\in R|f(gx)=\chi(g)f(x)$ for all $g\in G\}=\bigoplus_{\chi'-\chi\in L} k\cdot t^{\chi'}$. Each $Q_\chi$ has codimension 1 in $R_\chi$ because $R_\chi/Q_\chi$ is a 1-dimensional vector space (the generators of $Q_\chi$ identify all the basis vectors of $R_\chi$). If $J$ is some $G$-invariant ideal, then $J=\bigoplus_{\chi\in T^\vee/L} (J\cap R_\chi)$ \anton{present this better}. So the ideal $I^L$ is a \emph{maximal} $G$-invariant proper ideal in $R$ because any ideal that contains $R_\chi$ must be all of $R$ since $t^{\chi'}R_\chi=R_{\chi'+\chi}$. So we must have $I_G=I^L$.
 
 Finally, we'd like to show that $p\nmid |\Ga|$. This is equivalent to showing that $T^\vee/L$ doesn't have an element of order $p$. If there is an element of order $p$, then you can find $\chi\not\in L$ such that $\chi^p\in L$. But then $\chi^p-1\in I_G$, so $\chi-1\in I_G$ (because $I_G$ is radical (since $G$ is required to be a variety), so $\chi-1\in I_G$, so $\chi\in L$. \anton{more generally, if we don't require $G$ to be reduced, I think we get that $\Ga$ is a diagonalizable group scheme.}
\end{proof}
\begin{lemma}
 Suppose $G$ is a reductive group, then $Z(G)_s=Z(G)$.
\end{lemma}
\begin{proof}
 Suppose $u\in Z(G)_u$ with $u\neq 1$, and suppose $G\to GL(V)$ is a representation on which $u$ acts non-trivially. Then $\ker(u-1)$ is a $G$-invariant subspace of $V$, but it doesn't split because $(u-1)$ is a nilpotent operator.
\end{proof}
\begin{proposition}
 If $G$ is a reductive group, then $\g_s=\g$.
\end{proposition}
\begin{proof}
 We can assume $G$ is not abelian, because in the abelian case, it's already clear. Pick a representation $V$ of $G$. Consider $\sym^p(V)\supseteq W=\{u^p|u\in V\}$. $W$ is an invariant subspace.
 
 Claim: If $\g$ has a non-zero nilpotent element, then there is no $\g$-invariant subspace $W'$ such that $\sym^p V=W\oplus W'$. The action on $\sym^pV$ is $g(x_i)=\sum a_{ij} x_i\pder{}{x_j}$. So $W\subseteq (\sym^p V)^G$. Suppose we have a non-zero nilpotent element $A\in \g$. Pick $x,y$ such that $Ax=y$ and $Ay=0$. Then $Ax^p=0$ and $A(xy^{p-1})=y^p$. Any $W'$ must contain a vector of the form $xy^{p-1}+z^p$ \anton{because in the quotient $\sym^p(V)/W$, we have some element $xy^{p-1}$, so take a lift}, and when we apply $A$ to it, we get into $W$, so $W'$ is not invariant.
\end{proof}
\begin{proposition}
 Suppose $G$ is connected and $\g_s=\g$. Then $G$ is abelian (and must therefore be a torus).
\end{proposition}
\begin{proof}
 Let $G\subseteq GL(V)$. We will induct on the dimension of $G$ \emph{and} the dimension of $V$. The idea is to find a subgroup which is connected, so it must be a torus, and then procede.
 
 I will assume $k\neq \overline\FF_p$, but note that if $k=\overline \FF_p$, then you can change base to some transcendental extension and apply the proof. The conclusion is stable under base change. I want to have elements of infinite order, and elements in an algebraic group over $\overline \FF_p$ are of finite order.\footnote{Note that since $GL_n(\FF_q)$ is a finite group, all its elements are of finite order. It follows that all elements of $GL_n(\overline \FF_p)$ are of finite order, so all closed points of a group $G$ over $\overline \FF_p$ are of finite order.}
 
 \underline{Step 1.} $G$ has a dense set of elements of infinite order. If $g\in G$ is of finite order, its characteristic polynomial $p_g(t)\in \overline\FF_p[t]$ \anton{or $\overline\QQ[t]$ if in characteristic zero}. The coefficients are $\sigma_i(g)\in \overline\FF_p$, the elementary symmetric functions. Since $G$ is connected, it is impossible because the set of values of a regular function is a constructible set \anton{a constructible set contains an open subset of its closure, so the only dense constructible sets are those that contain an open subset, but $\overline\FF_p\subseteq \AA^1_k$ contains no open subset, so the only constructible subsets of $\overline\FF_p$ are finite sets.}. So each $\sigma_i(g)$ is constant, so every $g$ must have the same characteristic polynomial as the identity element of $G$, $(1-t)^n$, so all elements are unipotent. So the closed subgroup $G_u$ is equal to $G$, implying $\g=\g_u$, a contradiction. Density follows similarly \anton{you get that the $\sigma_i$ are constant on an open neighborhood of the identity}.
 
 \underline{Step 2.}
 \begin{lemma}
  For $s\in G_s$, let $C_G(s)=\{g\in G|sgs^{-1}=g\}$ and $C_\g(s)=\{x\in \g|\Ad_s(x)=x\}$. Then $\lie C_G(s)=C_\g(s)$.
 \end{lemma}
 \begin{proof}
  $\lie C_G(s)\subseteq C_\g(s)$ is clear, so we only need to show that the dimensions are equal. The result is true in the case $G=GL(n)$ because the condition of being the in the centralizer looks exactly the same in the group and the Lie algebra, and it is some linear condition $C_G(s)=\{g|gs=sg\}$ (even if $s$ is not semi-simple, btw). So all the $p$-th power problems don't appear here at all. \anton{$C_G(s)$ may be non-reduced in general, and the $GL_n$ argument works in general. The argument that follows is for $C_G(s)_\mathrm{red}$ rather than $C_G(s)$.}
  
  Now consider two locally-closed Zariski sets. $Y=\{gsg^{-1}s^{-1}|g\in G\}$ and $S=\{gsg^{-1}s^{-1}|g\in GL(n)\}$ (remember that $G\subseteq GL(n)$ is a closed subgroup). $Y\subseteq S\cap G$, so $T_eY\subseteq T_e S\cap \g$. Define $\nn=T_e S$, then $\gl=C_\gl(s)\oplus \nn$ \anton{$s$ is semisimple, so it always acts diagonalizably. Choose a basis where $\Ad_s$ is diagonal. Then $1-\Ad_s$ is some diagonal operator for which $T_eS$ is the image and $C_\gl(s)$ is the kernel, so you get a direct sum decomposition}. We have the decomposition $\g=C_\g(s)\oplus \m$ \anton{again using the diagonal $1-\Ad_s$}. Then $\m=\g\cap \nn$ \anton{$\g\subseteq \gl$ is a subspace invariant under the action of the diagonal operator $1-\Ad_s$}. This gives $\dim T_eY\le \dim \m$. But we also have $\dim T_e Y+\dim C_G(s)=\dim \g=\dim C_\g(s)+\dim \m$ by the usual dim group is dim orbit plus dim stabilizer. But this tells me that $\dim C_G(s)\ge \dim C_\g(s)$, which is what we wanted.
  
  The moral is that if an element is semi-simple, you get the same sort of thing you usually get in characteristic 0. 
 \end{proof}

\end{proof}