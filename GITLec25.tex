\sektion{25}{Lecture 25}

No class on Monday. I'll also be away Thanksgiving week, so there won't be class then.

Obtaining a toric variety as a quotient. Think of $\ZZ^s\subseteq \RR^s\supseteq \Sigma=\{\sigma\}$, where the $\sigma$ are rational polyhedral cones.

We pick $\bar\e_1,\dots, \bar\e_n\in \ZZ^s$ which are the generators of cones. Assume $\{\bar\e_1,\dots,\bar\e_n\}$ generate all of $\ZZ^s$ (as a lattice). For a cone $\sigma\in\Sigma$, we associate the set $J_\sigma=\{i|\bar\e_i$ is not a generator of $\sigma\}$. We define $V=\AA^n\setminus \{x^{J_\sigma}=0|\sigma\in \Sigma\}$ \anton{we're taking an intersection in that last set}.

\begin{claim}
 The toric variety $Y_\Sigma$ is a categorical quotient $V\quot T$, where $T$ is defined as below. Moreover, if $\Sigma$ is simplicial, then the quotient is a geometric quotient.
\end{claim}
We define an $s\times n$ matrix $A$ whose columns are $\bar\e_i$. We find an $n\times r$ matrix $B$ such that $AB=0$. Let $\chi_j$ be the rows of $B$ (the $\chi_j$ are a basis for the kernel of $A$). Then we define the action of an $r$-dimensional torus $T$ by $t\cdot (x_1,\dots, x_n)=(\chi_1(t)x_1,\dots, \chi_n(t)x_n)$.

The map $\ZZ^n\to \ZZ^s$ is surjective (by our assumption). The kernel is another free abelian group
\[
 0\to \ZZ^r\xrightarrow B \ZZ^n\xrightarrow A \ZZ^s\to 0.
\]
We regard these free abelian groups as lattices of 1-parameter subgroups of the tori in the short exact sequence
\[
 0\to T\to U\to Q\to 0.
\]
We also get the short exact sequence of character lattices
\[
 0\to Q^\vee\to U^\vee\to T^\vee\to 0
\]
where $Q^\vee$ is what we called $M$ before. Alsmost by construction, $k[M]=k[x_1^{\pm 1},\dots, x_n^{\pm 1}]^T$. I can write $V=\bigcup_{\sigma\in \Sigma}V_{x^{J_\sigma}}$. By construction, $V$ is covered by these open sets. What are invariant functions $\O(V_{x^{J_\sigma}})^G$? it is exactly $k[\sigma^\vee\cap M]$. Each of the affines is a categorical quotient, so when we glue them together, we get a categorical quotient.
\begin{remark}
 It may not be true that this is a Mumford quotient (exists a line bundle with linearization so that this is the semi-stable quotient). It is known that there are toric varieties that are proper but not projective, and Mumford quotients are always quasi-projective.
\end{remark}
\begin{remark}
 If we drop the assumption that the $\bar\e_i$ generate $\ZZ^s$, but generate a finite-index subgroup, then $Y_\Sigma$ is the quotient of $V$ by $T\times \Ga$, where $\Ga$ is some finite abelian group ($\Ga$ will be the cokernel of $A\colon \ZZ^n\to \ZZ^s$?). We get a morphism $U\to Q$, and the kernel of this morphism will contain $\Ga$ as a subgroup.
\end{remark}
\begin{example}
 \anton{$\bar\e_1=(0,1)$, $\bar\e_2=(1,0)$ and their negatives, with the upper left and lower right maximal cones missing} This is the fan of $\PP^1\times \PP^1$ minus two points. we have
 \[
  A=\Matx{1&0&-1&0\\ 0&1&0&-1}\qquad B=\Matx{1&0\\ 0&1\\ 1&0\\ 0&1}
 \]
 We have $(t,s)=(x_1,x_2,y_1,y_2)=(tx_1,sx_2,ty_1,sy_2)$, $\chi(t,s)=ts$. We have $V=\AA^4\setminus\{x_1x_2=0,y_1y_2=0\}$. The points where $x_1=y_1=0$ or $x_2=y_2=0$ are unstable, so they are always removed. We've also removed $x_1=y_2=0$ and $x_2=y_1=0$.
\end{example}
\begin{example}
 \anton{cone generated by $\bar\e_1=(2,1)$ and $\bar\e_2=(1,2)$} We get $\ZZ^2/\<\bar\e_1,\bar\e_2\>=\ZZ/3$. The acting torus is actually trivial here. The map $U\to Q$ is given by $(t_1,t_2)\mapsto(t_1^2t_2,t_1t_2^2)$. This is surjective \anton{}, with kernel $\{e,(\e,\e),(\e^2,\e^2)\}$ where $\e=\sqrt[3]{1}$. The quotient is $\AA^2/(\ZZ/3)$.
\end{example}

\subsektion{Chow quotient and Hilbert quotient}
Here's a way to compactify quotients.

Idea: Suppose $X$ is projective and irreducible, and $G$ acts on $X$. The main problem with the quotient is that there is some open space of good orbits (e.g.~stable with respect to some linearization). So in general, there exists some Zariski open set $U\subseteq X$ with the follwoing property. For $x\in U$, $\bbar{G\cdot x}$ is a closed subvariety of $X$, so we can regard it as an algebraic cycle (say its dimension is $r$). For $x,y\in U$, the closures of the orbits will be rationally equivalent cycles, so they represent the same homology class in $H_r(X,\ZZ)$ \anton{we'll prove that there is a $U$ with this property later}.

$U\quot G\hookrightarrow C_r(X,\delta)$, the Chow variety (the variety of algebraic cycles with cohomology class $\delta$), where $\delta\in H_r(X,\ZZ)$. If you like, you can think about $U\quot G$ as being in $\hilb_r(X)$.
\begin{definition}
 The \emph{Chow quotient} $X\quot_C G$ is the closure of $U\quot G$ in $C_r(X,\delta)$. %The \emph{Hilbert quotient} is the closure of $U\quot G$ in $\hilb_r(X)$.
\end{definition}
Note that no linearization is involved so far.

Realization of the Chow variety. Consider $C_r(\PP^n,d)$, the variety of subvarieties of $\PP^n$ of dimension $r$ and degree $d$. Then we'll restrict to $X$. Classically, this was understood with \emph{Chow forms}. Pick $r+1$ linear forms $\ell_0,\dots, \ell_r$ on $\PP^n$. For an algebraic cycle $Z=\sum c_i Z_i$, there exists a polynomial $R_Z(\ell_0,\dots, \ell_r)$ with the following properties. $(*)$ $R_{Z_1+Z_2}=R_{Z_1}R_{Z_2}$, and $(**)$ If $Z$ is an irreducible variety, $R_Z(\ell_0,\dots, \ell_r)=0$ if and only if $Z\cap \{\ell_0,\dots, \ell_r=0\}\neq\varnothing$.

$C_r(\PP^n,d)$ is the projectivization of the space of homogeneous polynomials in $\ell_0,\dots, \ell_r$ of degree $d$.

\begin{example}
 Consider two points in $\PP^2$, with coordinates $(x_1,y_1,z_1)$ and $(x_2,y_2,z_2)$. A form is of the form $ax+by+cz$. The form $R_Z$ will be $(ax_1+by_1+cz_1)(ax_2+by_2+cz_2)$, which is in the space of quadratic forms in $a,b,c$. So we are interested in those forms which are the product of two linear forms. This is the condition that it's rank is at most 2. $C_0(\PP^2,2)$ is then the projectivization of the space of quadratic forms with zero discriminant, which is of dimension 4. This is singular when the rank drops to 1, which corresponds to the case when two points coincide. The space of forms of rank 1 is a $\PP^2$ embedded by the Veronese map.
\end{example}

If you do the same with Hilbert schemes, what will happen? The Hilbert scheme and the Chow variety will agree when two points do not coincide, but they will be different on the degenerate locus.

Consider $\AA^2\subseteq \PP^2$. You are looking for ideals $I\subseteq k[x,y]$ such that the quotient is 2-dimensional. If two points don't coincide, then this ideal is the intersection of two maximal ideals. If the two points coincide, then $I$ cannot be the square of a maximal ideal (the quotient would be 3-dimensional), so the Hilbert scheme records the "direction of collision", whereas the Chow variety doesn't see this. So the Hilbert scheme is non-singular, whereas the Chow variety was singular. The Hilbert scheme in this case is exactly the resolution of the singularities of the Chow variety. In general, there is a map from Hilbert to Chow.

Next we'll show that for any Mumford quotient, you get a map to the Chow quotient. The Chow quotients are difficult to compute in general, but it has been done in the case of toric varieties, for example. This Chow quotient is not a categorical quotient, which is bad, but there are many good sides too.
