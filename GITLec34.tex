\sektion{34}{Chow quotients in the toric case}
\renewcommand{\A}{\mathcal A}

I put some exercises on the web which you can do while I'm away.

Recall that we have the situation that $X$ is a projective toric variety, and we consider the quotient $X\quot H$, where $H\subseteq T$ is a subtorus. We have the map $\pi$ from $P\subseteq \hhat T_\RR$ to $Q\subseteq \hhat H_\RR$.

We first define the \emph{fiber polytope} $\Sigma(P,Q)$. We define a volume form $dqq$ on $\hhat H_\RR$ (such that the volume of the lattice parallelopiped is $1$). Let $s\colon Q\to P$ be a continuous section, then we get $I_s=\int_Q s(q)\,dq\in \hhat T_\RR$. Define $\Sigma(P,Q)=\{I_s|s$ a continuous section$\}$. This polytope is contained in the fiber over the center of mass of $Q$. The main result is that this fiber polytope is the polytope of the Chow quotient.
\begin{proposition}\label{lec34P:ChowPolytope}
 The moment polytope of $X\quot_C H$ is $\Sigma(P,Q)$.
\end{proposition}
\begin{proposition}[Billera-Sturmfels]
 The dual fan to $\Sigma(P,Q)$ is the common refinement of all dual fans to polytopes that arrise as fibers of $\pi\colon P\to Q$.
\end{proposition}
For the proof of \ref{lec34P:ChowPolytope}, we can restrict to the case $X=\PP^n$ and $T=(\CC^\times)^{n+1}$. In general, we have $\Delta_n\xrightarrow\ga P\xrightarrow\pi Q$, and the notion of fiber polytope is well-behaved. So we may assume $\Delta_n=P$.

$\pi\colon \Delta_n\to Q$. We have the corners $\e_0,\dots, \e_n$, which are mapped to $\pi(\e_i)=\chi_i$; let $\A=\{\chi_i\}$. Some of these are vertices of $Q$, and some are interior points. Look at all triangulations of $Q$ with vertices in $\A$ (not all elements of $\A$ need to be involved). For a triangulation $\tau$, consider $\phi_\tau = \sum_{\sigma\in \tau} vol(\sigma(i_0,\dots, i_k))(\e_{i_0}+\cdots +\e_{i_k})$.
\begin{claim}
 $\Sigma(P,Q)$ is the convex hull of the $\phi_\tau$.
\end{claim}
The idea is that the extreme sections give you points on the vertices of your fiber polytope, and triangulations give you these extremal sections.

\medskip

\underline{Chow variety}. Let $Z\subseteq \PP^n$ be an irreducible subvariety of dimension $k$. Consider $H_Z=\{L\in Gr(n+1,n-k)|\PP(L)\cap Z\neq\varnothing\}$. If $Z$ is irreducible, this $H_Z$ is an irreducible hypersurface in $Gr(n+1,n-k)$ \anton{exercise}. The equation $R_Z(\ell_0,\dots, \ell_k)=0$ giving $H_Z$ is called the \emph{Chow form}. If you have a cycle, the chow forms multiply. Think of $L$ as given by $\ell_0=\cdots =\ell_k=0$ (the $\ell_i$ are linear forms). Since $R_Z$ is a form on the grassmannian, it is a form in the minors of the corresponding matrix, with coefficients $\ell_i^j$. In other words, $R_Z$ depends only on the pl\"uker coordinates, and of course it only is defined up to scalar, so we consider $R_Z\in \PP(\sym^d \bigwedge^{k+1}(\CC^{n+1}))$.

The group $GL(n+1)$ (and it's maximal torus $T$) acts on $\CC^{n+1}$, so it acts on $\PP(\sym^d \bigwedge^{k+1}(\CC^{n+1}))$. This action is compatible with the action on something.

To find the polytope of $\PP^n\quot H$, pick a generic point (like $[1:\cdots :1]$), and let $Z$ be the closure of the $H$-orbit. The Chow quotient $\PP^n\quot H$ will be a toric variety with the torus $T/H$. Consider $\overline{T\cdot R_Z}$. The vertices of the corresponding polytope $\mu(\overline{T\cdot R_Z})$ are $T$-weights of the representation $\sym^d(\bigwedge^{k+1}(\CC^{n+1}))$, so it's clear that the have the same weights as $\phi_\tau$. The weights are of the form $\sum_{i_0<\cdots <i_k}m_{i_0\dots i_k}(\e_{i_0}+\cdots +\e_{i_k})$, with $\sum m_{i_0\dots i_k}=d$, so combinatorially what I said makes perfect sense.

How to actually find $R_Z$? In general, you can use the Koszul complex. Fix some $r\gg 0$. Let $Z$ be an irreducible projective variety. The Koszul complex is $K_r^i(\ell_0,\dots, \ell_k)=\Ga(\O_Z(i+r))\otimes \bigwedge^i(\CC^{k+1})$. Fix a basis $e_0,\dots, e_k$ for $\CC^{k+1}$. The differential will be given by $\partial (f\otimes \om) = \sum \ell_if\otimes e_i\wedge \om$. It turns out that $K^\udot_r(\ell_0,\dots, \ell_k)$ is exact if and only if $L\in H_Z$. To see this, consider the corresponding complex of sheaves (don't take global sections), which is the complex of forms, and the differential is wedging with the form $\sum \ell_i e_i$. If $L\not\in H_Z$, this form is not zero at any point of the variety, so locally the complex is exact because it is just muliplication by a 1-form. If there is a point where the form vanishes, then it's clear you get cohomology. The idea is to make $r$ large enough that the sheaf is generated by global sections. This is a finite-dimensional complex with finite lenght. Generically, the complex is exact. You can measure this using the \emph{determinant} of the complex. For the complex $0\to A\xrightarrow\partial B\to 0$, you get $\det\partial \in \bigwedge^{\text{top}} A^*\otimes \bigwedge^{\text{top}}B$. In general, $\det \partial \in \bigwedge^{\text{top}} (K^0)^* \otimes \bigwedge^{\text{top}} (K^1)\otimes \bigwedge^{\text{top}} (K^2)^* \cdots $. \anton{somehow the idea is to divide by the images of the maps. Locally, you write the matrices, take a suitable minor, and take the first minor, divide by the determinant of the next minor, then multiply by the determinant of the next minor, and so on. This is a functorial ratio of minors}. The complex is not exact if and only if you have a pole or a zero. From this, it's clear that this hypersurface $H_Z$ is either the set of poles or the set of zeros of $\det \partial$. If you want, you can look at the paper. The main claim is
\begin{claim}
 $R_Z(\ell_0,\dots, \ell_k)=\det(K^\udot_r(\ell_0,\dots, \ell_k))^{(-1)^{k+1}}$.
\end{claim}
Now you have a formula for computing the Chow form.

Now I take 1-parameter subgroups in $T$, written as $t^\lambda$. $\lambda\colon \A\to \ZZ$ you can assume the weights are positive because you're working in projective space. A regular triangulation is one so that there exists a strictly convex function $Q\to \RR$ which is linear on simplices $\sigma\in \tau$. The height at each vertex is $\lambda_i$. So to each $t^\lambda$, we can associate a special triangulation. It's not hard to see that in the fiber polytope, only the regular triangulations give you vertices ... the others give you interior points. $\lim_{t\to \infty}t^\lambda Z=\sum vol(\sigma)L_\sigma$, where $L_\sigma=span\{e_{i_0},\dots, e_{i_k}\}$ is the span of the vertices of $\sigma$. To do this, you should study the complex $K^\udot_r(t^\lambda\ell_0,\dots, t^\lambda \ell_k)$. There is a filtration, and it is possible to calculate the associated graded complex. The limit will be something like $\lim_{t\to \infty} t^\lambda R_Z = \prod (\sigma\text{-minor})^{vol(\sigma)}$. This is a technical thing done in the work of Koushinerenko. If you want to show the degree of the cycle $Z$, it is clearly the volume of $Q$ because the degree is the sum of the volumes.
\begin{corollary}
 $\deg(\overline{H\cdot [1:\cdots :1]})=vol(Q)$.
\end{corollary}
Part of this stuff I'll put in the exercises.

After the break, I'd like to talk about Luna's slice theorem and stable vector bundles.

No class Nov.~20,23,25, but we'll have classes Nov.~30, Dec.~2, 4, 7, 9.