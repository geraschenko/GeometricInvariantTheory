\sektion{15}{The Hilbert-Mumford numerical criterion}

\begin{remark}
 The quotient we construct \emph{depends} on the embedding $X\to \PP(V)$ because the notion of stability depends on the embedding.
\end{remark}

\begin{proposition}
 Let $T$ be a torus of dimension $n$ acting on $\AA^N$, with the stabilizer of a generic point being finite. Let the action be given by $t\cdot (x_1,\dots, x_N)=(\chi_1(t)x_1,\dots, \chi_N(t)x_N)$. We regard the $\chi_i$ as vectors in $T^\vee\cong \ZZ^n\subseteq \QQ^n\subseteq \RR^n$. We defined $\supp(x)=\{\chi_i|x_i\neq 0\}=\{\chi_1,\dots, \chi_k\}$
 
 $x$ is semi-stable if $0\in C(\supp(x))$ and it is stable if $0$ is an interior point.
\end{proposition}
\begin{proof}
 Suppose $0\not\in C(\chi_1,\dots, \chi_k)$, then there exists a $\lambda\in (\RR^N)^*$ such that $\<\lambda,\chi_i\>>0$ for $1\le i\le k$. In fact, since there are finitely many $\chi_i$, we may assume $\lambda\in (\QQ^N)^*$, so after rescaling, we may assume $\lambda\in (\ZZ^N)^*=(T^\vee)^*$. So $\lambda$ defines a 1-parameter subgroup $\lambda\colon k^\times\to T$.
 
 Now we consider $\lim_{t\to 0} \lambda(t)x$. By the condition that $\lambda$ is positive on the $\chi_i$, we have that $\lambda(t)x=(t^{a_1}x_1,\dots, t^{a_N}x_N)$ for strictly positive $a_i$ \anton{at least for those $i$ for which $x_i\neq 0$}. So the limit is zero, so $0\in \bbar{T\cdot x}$, so $x$ is unstable.
 
 If $0\in C(\chi_1,\dots, \chi_k)$, then we can choose $a_1,\dots, a_k\in \RR_{\ge 0}$ not all zero such that $\sum a_i \chi_i=0$. Consider the exact sequence
 \[\xymatrix@R=0pc @C=1pc{
  0\ar[r]& K\ar[r]& \QQ^k\ar[r]& T^\vee\otimes\QQ\\
  & & (a_1,\dots, a_k)\ar@{|->}[r] & \sum a_i\chi_i
 }\]
 Since $\RR$ is flat over $\QQ$, we get an exact sequence
 \[
  0\to K\otimes \RR\to \RR^k\to T^\vee\otimes \RR.
 \]
 By assumption, we have an element of $K\otimes \RR$ for which all the $a_i$ are non-negative and not all zero. Since $K$ is dense in $K\otimes \RR$, we can find \emph{rational} non-negative $a_i$ (not all zero) such that $\sum a_i\chi_i=0$. After scaling, we may assume they are integers. Then the function $f(x)=x_1^{a_1}\cdots x_k^{a_k}$ is $T$-invariant and non-zero at $x$ (by the definition of $\supp(x)$). So $x$ is semi-stable.
 
 Note that the dimension of the span of $\{\chi_1,\dots, \chi_k\}$ is $n$ if and only if the stabilizer of $x$ is finite. This is because a 1-parameter subgroup $\lambda$ is in the stabilizer of $x$ if and only if $\{\chi_1,\dots, \chi_k\}\subseteq \lambda^\perp$, and a subgroup of the torus is finite if and only if it contains no non-trivial 1-parameter subgroups.
 
 If $0$ is in the interior of $C(\chi_1,\dots, \chi_n)$, then the dimension of the span of the $\chi_i$ must be $n$; otherwise, $C(\chi_1,\dots, \chi_k)$ would have no interior. So the stabilizer of $x$ must be finite. We can choose the $a_i$ to be \emph{strictly} positive integers. Then $V_f=\{x|$all $x_i\neq 0\}$. Then every point in $V_f$ has the same support as $x$, so every point in $V_f$ has finite stabilizer, so all points are regular, so the action is closed on $V_f$.
\end{proof}
This result allows you to draw pictures to find out which points are stable and which are not.

We have the following corollary, which says that stability can be checked on 1-parameter subgroups. If $x$ is unstable, then we may find the 1-parameter subgroup $\lambda\colon k^\times\to G$ which demonstrates instability. On the other hand, in the other cases (when $x$ is (semi)-stable) I can't find such a 1-parameter subgroup.
\begin{corollary}
 In the case $G=T$, proper stability can be checked on 1-parameter subgroups.
\end{corollary}
\begin{theorem}[Hilbert-Mumford numerical criterion]
 If $G$ is reductive, then 
 \begin{enumerate}
  \item $x$ is semi-stable if and only if it is semi-stable for any 1-parameter subgroup, and
  \item $x$ is properly stable if and only if it is properly stable for any 1-parameter subgroup.
 \end{enumerate}
\end{theorem}
Semi-stable for a 1-parameter subgroup means that 0 is not in the closure. Stable means that the orbit is closed and the stabilizer is finite.

For any reductive group, you get a maximal torus. All maximal tori are conjugate, and any 1-parameter subgroup lies in a maximal torus. Checking stability for all 1-parameter subgroups is annoying, but checking it for maximal tori is easier.

I used \cite{git}, and there is one statement that isn't completely clear to me. Other people do it differently.
\begin{lemma}
 For $x\in V$, suppose $y\in \bbar{G\cdot x}\setminus G\cdot x$. Let $O=k[\![t]\!]$ and $K=k(\!(t)\!)$, with $\m=(t)\subseteq O$. Then there exists $\ga\in G(K)$ such that $\ga\cdot x\in O\otimes V$ and $\ga\cdot x\equiv y$ modulo $\m$.
\end{lemma}
$\ga(t)\in G$ is a laurent series. When I apply it to $x$, I get a formal power series $\ga(t)\cdot x\in k[\![t]\!]V$, and $\lim_{t\to 0} \ga(t)x=y$.
\begin{proof}
 Let $Z=\bbar{G\cdot x}$. Consider $\phi\colon G\to Z$ given by $g\mapsto g\cdot x$. This is a dominant open immersion (recall that the orbit is open in its closure \anton{ref}). I claim that there is a curve $C\subseteq Z$ such that $y\in C$ and $C\cap G\cdot x\neq\varnothing$ \anton{this is like checking valuative criteria on an open set}. Suppose $Z\setminus G\cdot x\subseteq \{h(x)=0\}$. If $\dim \m_y/\m_y^2 >1$, then there is an $f\in \m_y$ such that $f\not\equiv h$ modulo $\m_y^2$. Take $\{f(x)=0\}$. Keep decreasing the dimension to get a curve.
 
 We may take some curve $C_1$ whose image is in $C$, so we have
 \[\xymatrix{
  \llap{$s\in\;$}\bbar C_1 \ar[d]_{\phi} & C_1\ar[l]\ar@{^(->}[r]\ar[d]_{\phi|_{C_1}} & G\ar[d]^\phi\\
  \llap{$y\in\;$}\bbar C & C\ar[l]\ar@{^(->}[r] & Z
 }\]
 The map $\phi|_{C_1}$ is not surjective, but it is dominant. We can take the projective completions of the affine curves $C$ and $C_1$ and extend the map. Let $s\in \bbar C_1$ be a point that maps to $y$. We may assume $\bbar C_1$ is non-singular at $s$ by taking normalization if needed. Let $U$ be an open neighborhood of $s$. We have $\ga\colon U\setminus s\to G\xrightarrow{\phi}C$. $\ga$ has a pole at $s$. $\phi(\ga(U))\subseteq C$, $\phi(s)=y$. \anton{once you go from $G$ to $C$, the pole goes away}

 Let $O_s$ be the local ring of $s\in \bbar C_1$, and $K_s$ the field of fractions. We've constructed $\ga\in G(K_s)$ and the condition at the end of the previous paragraph means that $\ga\cdot x\in O_s\otimes_k V$ and $\ga\cdot x\equiv y$ modulo $\m_s$.
 
 Now we just complete with respect to the maximal ideal and since $s$ was a smooth point, you get $k[\![t]\!]$.
\end{proof}
The next result I'll only prove for $SL(n)$, because it involves some structure theory of semi-simple groups.
\begin{theorem}[Iwahori]
 Let $G$ be a reductive algebraic group. Then each double coset in $G(O)\backslash G(K)/G(O)$ contains a (unique!\anton{probably}) 1-parameter subgroup $\lambda(t)$.
\end{theorem}
Consider the case of $SL(n)$. Start with a matrix $X=(x_{ij})\in G(K)$, so each $x_{ij}$ is a Laurent series. We have a valuation (the smallest power of $t$ that appears; the order of the zero/pole). Choose $v(x_{ij})$ minimal. By multiplying on the left and right by permutation matrices, we may assume $(i,j)=(1,1)$, so $v(x_{11})$ is minimal. Multiplying $X$ on the left by some matrix, we can do Gaussian elimantion (we can divide because we chose $v(x_{11})$ minimal) to get the first column to be all zeros below $x_{11}$. Similarly, we can get all zeros to the right of $x_{11}$. Now repeat on the smaller matrix until we get a diagonal matrix by induction. So we may assume we have the matrix $diag(z_{11},\dots, z_{nn})$. Since we are in $SL(n)$, the product of the $z_{ii}$ is 1. Each $z_{ii}$ is of the form $t^{a_i}\cdot f_i$ where the $f_i\in k[\![t]\!]=O$ are invertible. So we have $\sum a_i=0$ and $\prod f_i=1$. So we multiply our matrix on the left by $diag(f_1^{-1},\dots, f_n^{-1})$ to get our 1-parameter subgroup.
