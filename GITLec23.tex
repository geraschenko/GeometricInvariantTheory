\sektion{23}{Lecture 23}

If the group $G$ acts on $\AA^n$, then a linearized line bundle is determined by a choice of character. The action on the line bundle (which must be trivial) is given by $g\cdot (f(x),x)=(\chi(g)f(x),g\cdot x)$, where $f$ is a regular function. So the invariant sections are those regular functions $f$ for which $\chi(g)f(x)=f(g\cdot x)$, the semi-invariants with character $\chi$. The invariant sections of the $n$-th tensor power are semi-invariants with character $\chi^n$. The proj quotient is $\proj \bigoplus_{n\ge 0} \Ga(\AA^n,\O_X^{\otimes n})^G$.

\begin{example}
 Consider matrices $\matx{x_1&x_2&x_3\\ y_1&y_2&y_3}$ modulo the action of diagonal matrices on the left and right (call this torus $T$). Each coordinate has a weight with respect to the torus action. If I draw them all together, I get the picture \anton{prism on a triangle with $x$'s on top and $y$'s on bottom. actually a 4-dimensional picture with the prism in a plane (not through zero)}
 
 The quotient depends on the choice of a character $\chi=(a_1,a_2|b_1,b_2,b_3)$. For example, the character of the action on $x_1$ is $(1,0|1,0,0)$. That is, $x_1$ is a semi-invariant function with character $(1,0|1,0,0)$. Even though $T$ is 5-dimensional, the action factors through a 4-dimensional torus, but these coordinates are convinient. All the coordinates have characters that lie on one side of a plane, so the quotient is projective \anton{The proj quotient is always projective over $R^G=\Ga(\AA^n,\O^{\otimes 0})^G$, the degree zero part of the graded ring that defines the proj quotient. I think this comment means that $R^G=k$, so the quotient will always be projective over $\spec k$.}.
 
 If $\chi=0$, then the quotient is a point.
 
 If I take $\chi=(3,0|1,1,1)$. Then the quotient is $\proj k[x_1x_2x_3]$ \anton{really $u^3x_1x_2x_3$, but we'll ignore the $u$'s because that's just to get the grading}.
 
 In general, we get the invariants $R^\chi=\bigoplus R^\chi_d$ where $R^\chi_d=\{f\in k[\AA^n]|f(tx)=\chi^d(t)f(x)\}$, the semi-invariants with character $\chi$.
 
 Suppose $\chi=(1,1|1,1,0)$. Then we get $\proj k[x_1y_2,x_2y_1]$, which is $\PP^1$. We've increased the dimension, but we'd really like to get a 2-dimensional quotient because the orbits are 4-dimensional.
 
 Let $\chi_1=(2,1|1,1,1)$. Then invariants are $\alpha_1=y_1x_2x_3$, $\alpha_2=x_1y_2x_3$, and $\alpha_3=x_1x_2y_3$, which generate all invariants. The quotient is $X\quot_{L_\chi} T = \proj k[\alpha_1,\alpha_2,\alpha_3]=\PP^2=Y_1$. Consider the open set where all the $x_i$ are non-zero. Then by the action of the torus, we can make $x_1=x_2=x_3=1$. Then we have $y_i=\alpha_i$, and we can multiply the second row by any unit $t_2$, so we get a copy $\PP^2$. Note that if all the $\alpha_i=0$, then the point is unstable because we can get the orbit to have 0 in its closure. We have three special lines, given by $\alpha_i=0$ for $i=1,2,3$.
 
 If $\alpha_2=\alpha_3=0$, then we have the orbit corresponding to $\matx{1&1&1\\ *&0&0}$. But we see that if we evaluate the $\alpha_i$ on $\matx{0&*&*\\ *&*&*}$, we get $\alpha_2=\alpha_3=0$, so this whole set gets sent to a single point in $\PP^2$.
 
 Let $\chi_2=(1,2|1,1,1)$. As before, we get three invariants $\beta_3=y_1y_2x_3$, $\beta_2=y_1x_2y_3$, $\beta_1=x_1y_2y_3$, with the identification $\matx{\beta_1&\beta_2&\beta_3\\ 1&1&1}$. We have the identification $\beta_i=1/\alpha_i$. Except we see that over the point $P$ we found in the previous paragraph, we get the line $\beta_1=0$. That is, every element of the form $\matx{0&*&*\\ *&*&*}$ gives you a different point on the line $\beta_1=0$. Call the quotient $Y_2\cong \PP^2$.
 
 Now let's add these characters together to consider the case $\chi_3=(3,3|2,2,2)$. Here there are plenty of semi-invariants. $z=x_1x_2x_3y_1y_2y_3$, $v_1=x_1^2x_2y_2y_3^2$, $v_2=y_1^2x_2y_2x_3^2$, $w_1=x_1y_1y_2^2x_3^2$, $w_2=x_1y_2x_2^2y_3^2$, $u_1=y_1^2x_2^2x_3y_3$, and $u_2=x_1^2y_2^2x_3y_3$. The proj of the ring generated by these is a surface in $\PP^6$. The relations are given by $z^2=v_1v_2=w_1w_2=u_1u_2$ and $z^3=u_1v_1w_1=u_2v_2w_2$. Call the quotient $Y_3$.
 
 You can see that $X^{ss}(L_{\chi_3})$ is the intersection of the semi-stable points of $\chi_1$ and $\chi_2$. The set of semi-stable points drops, and the size of the quotient goes up, which is what normally happens.
 
 Outside of the three special lines on $Y_1$ and $Y_2$, we get that the two are isomorphic. Consider the closure of the rational map from $Y_1$ to $Y_2$, which is a surface in $\PP^2\times\PP^2$. This surface will be $Y_3$. Over any point other than the three distinguished points, you get a unique point, and over the special points, you get a $\PP^1$. To see that, you can glue out of two pieces $\matx{0&1&1\\ 1&\alpha_2&\alpha_3}$ and $\matx{0&\beta_1&\beta_2\\ 1&1&1}$.
 
 The upshot is that $Y_3$ is the blowup of $Y_1$ at the three special points. $Y_3$ is a del Pezzo surface. In this case, I believe this is ``the biggest'' quotient. In general, there need not be a biggest one, like in the example from last time.
 
 You can get the same surface if you blow up two points on $\PP^1\times \PP^1$. Perhaps for a suitable choice of character, you can get the quotient $\PP^1\times\PP^1$.
\end{example}

\subsektion{General \texorpdfstring{$\AA^n\quot T$}{An//T}}

Fix a torus $T$ of dimension $r$ acting on $\AA^n=\{(x_1,\dots, x_n)|x_i\in k\}$, given by $t\cdot (x_1,\dots, x_n)=(\chi_1(t)x_1,\dots, \chi_n(t)x_n)$, with $\chi_1,\dots, \chi_n\in T^\vee$. Let $a\in \ZZ^n$ be $a=(a_1,\dots, a_n)$, and define $x^a=x_1^{a_1}\cdots x_n^{a_n}$. I'm interested in monomials because all the semi-invariants will be monomials. We define $\supp(a)=\{i|a_i\neq 0\}$. Given $I\subseteq \{1,\dots, n\}$, we define $x^I=\prod_{i\in I}x_i$. We define $M=\{a\in \ZZ^n|\sum a_i\chi_i=0\}\subseteq \ZZ^n\subseteq \RR^n$. We can define the group-algebra of $M$, which will be a subring of Laurent polynomials in $t_1,\dots, t_n$. More generally, for a submonoid $S\subseteq \ZZ^n$, we define $k[S]$ to be the ring $k[S]=\{x^a|a\in S\}$.

Fix a character $\chi$. Let's describe the quotient $\AA^n\quot_{L_\chi}T$. In our case, $S_k=\{a\in\ZZ^n_{\ge 0}|\sum a_i\chi_i=k\chi\}$. Then $k[S_k]$ is the space of semi-invariants corresponding to the character $k\chi$, and $k[S_0]$ is exactly the ring of $T$-invariant polynomials. Let $S=\bigcup_{k\ge 0} S_k$ and consider the ring $k[S]=\bigoplus k[S_k]$.

To construct the quotient, I'll glue it out of $X_f$'s as always. I have the ideal $k[S]_{>0}\subseteq k[S]$ generated by positive degree monomials. Choose a minimal set of monomial generators $\{f_1,\dots, f_p\}$ of this ideal. Define $U_i=\{x\in \AA^n|f_i(x)\neq 0\}$. These are all semi-stable points by definition, and the $U_i$ cover $X^{ss}$. I define $R_i=\{g/f_i^k|\deg g=k\deg f_i\}$.

Each $f_i$ is a monomial, so $f_i=x^{m_i}$. Let $J_i=\supp m_i$. Then $U_i=\{x\in \AA^n|x_j\neq 0$ for all $j\in J_i\}$. Define $\sigma_i^\vee=\{a\in \RR^n|a_j\ge 0$ for all $j\not\in J_i\}$.
\begin{claim}
 $R_i=k[\sigma_i^\vee\cap M]$.
\end{claim}
\begin{proof}
 It is clear that any $g/f_i^k\in k[M]$. But for any element $h\in \sigma_i^\vee\cap M$, by multiplying by a sufficiently high power of $f_i$, I can make all the exponents for coordinates in $J_i$ non-negative. Say $hf_i^k$ has positive exponents for all coordinates in $J_i$ are non-negative.
\end{proof}
We have the inclusion $M\subseteq \ZZ^n$. We get the dual map $(\ZZ^n)^*\to M^*$. Next time we'll talk about fans.