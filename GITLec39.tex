\sektion{39}{Lecture 39}

Assume $G$ is geometrically reductive and acts on some ring $R$. Suppose $I\subseteq R$ is a $G$-invariant ideal. \anton{when $G$ was reductive, we had $R^G/I^G\cong (R/I)^G$} Then $R^G/I^G\subseteq (R/I)^G$.
\begin{lemma}
 If $r\in (R/I)^G$ then for some $m$, $r^m\in R^G/I^G$.
\end{lemma}
\begin{proof}
 $r$ is contained in some finite-dimensional $G$-invariant subspace $V\subseteq R$. So we get the exact sequence
 \[
  0\to W=I\cap V\to V\to kr\to 0
 \]
 where $W\subseteq V$ is invariant and $kr$ is the trivial representation. As we showed last time, if we take a symmetric power, the sequence splits:
 \[
  0\to \sym^{m-1}(W)V\to \sym^m(V)\to kr^m\to 0
 \]
 splits.
\end{proof}
Now we'll prove the first theorem from the end of the last lecture.
\begin{remark}
 Suppose $(R/I)^G$ is finitely generated. Then by the lemma, $R^G/I^G\subseteq (R/I)^G$ is an integral extension of rings, so $R^G/I^G$ is finitely generated as well.
\end{remark}
We'll prove the theorem first for graded rings (like with Hilbert's theorem), and use induction on Krull dimension.

\underline{Step 1}. $R=\bigoplus_{i\ge 0} R_i$ graded with $R_0=k$. We induct on the Krull dimension. Consider the ideal $R\cdot R^G_{>0}$. Since $R$ is noetherian, we can choose finitely many generators for the ideal (which we may assume are invariant) $f_1,\dots, f_k$. Let $f=f_1$. By induction on Krull dimension, we may assume $(R/Rf)^G$ is finitely generated. There are two cases: either $f$ is a zero-divisor or it's not.

(a) Suppose $f$ is not a zero-divisor. Then $(Rf)^G=R^Gf$. Pick $\alpha_1,\dots, \alpha_s$ representatives of generators for $R^G/(Rf)^G$ in $R^G$. Then $f,\alpha_1,\dots, \alpha_s$ generate $R^G$. To see this, given $x\in R^G$, we have $x=(\alpha_1,\dots, \alpha_k)+fy$, with $\deg y<\deg x$; now induct on degree of $x$.

(b) Suppose $f$ is a zero-divisor, then $J=\mathrm{Ann}(f)$ is a $G$-invariant ideal. Then $R^G/(Rf)^G$ and $R^G/J^G$ are finitely generated by the inductive hypothesis (and induction on number of generators). Pick $\alpha_1,\dots, \alpha_p$ representatives of generators of $R^G/(Rf)^G$ in $R^G$ and $\beta_1,\dots, \beta_q$ representatives of generators of $R^G/J^G$. Now consider $B=k[\alpha_1,\dots, \alpha_p,\beta_1,\dots, \beta_q]$. Let $c_1,\dots, c_n$ be representatives of generators of $(R/J)^G$ as a $B/B\cap J$-module. Then $fc_i\in R^G$ because $f(gc_i-c_i)=0$ for all $g\in G$.
\begin{claim}
 $R^G=B[fc_1,\dots, fc_n]$
\end{claim}
Let $x\in R^G$, then there is a $b\in B$ such that $x-b\in fR$ because the natural map $B\to R^G/(fR)^G$ is surjective. So $x-b=fc$, $fc$ is $G$-invariant so $c\in (R/J)^G$. So $x=b+fc\in B[fc_1,\dots, fc_n]$.

Note that for this argument (b), we did not use the assumption that $R$ is graded.

\underline{Step 2}. General case. Let $A=k[x_1,\dots, x_n]$ with $R=A/I$ for some $G$-invariant ideal $I$. I know that $A^G/I^G$ is finitely generated by the graded case. We have $A^G/I^G\subseteq (A/I)^G$ is an integral extension. If $R^G$ is an integral domain, then it is sufficient to check that the field of fractions $Q((A/I)^G)$ if finitely generated over $Q(A^G/I^G)$. If it is not an integral domain, then I can use the same argument as in part (b) of Step 1.

\bigskip

Now let's prove the second theorem from the end of last lecture (the separation lemma). We have invariant ideals $I(Z_1)$ and $I(Z_2)$ such that $I(Z_1)+I(Z_2)=k[X]$. Let $1=\alpha+\beta$ with $\alpha\in I(Z_1)$ and $\beta\in I(Z_2)$. Then $\beta(Z_1)=1$ and $\beta(Z_2)=0$, but $\beta$ is not invariant. There is a $G$-invariant finite-dimensional subspace $V\subseteq k[X]$ containing $\beta$. Let $\phi_1,\dots,\phi_n$ be a basis for $V$. We may assume $\phi_i=g_i\cdot\beta$ for some $g_i\in G$. We get a map $\phi\colon X\to \AA^n$. By invariance of $Z_1$ and $Z_2$, we get that $\phi(Z_1)=(1,\dots, 1)$ and $\phi(Z_2)=(0,\dots, 0)$. By geometric reductivity, there is a $G$-invariant homogeneous polynomial $F(\phi_1,\dots, \phi_n)$ such that $F(Z_1)=1$ and $F(Z_2)=0$.

\bigskip

In one book I used, I found this theorem. There is a third notion of a reductive group. A group $G$ is \emph{algebraically reductive} if its unipotent radical is trivial. The unipotent radical is the maximal normal unipotent subgroup, which is the intersection of the kernels of all irreducible representations. In characteristic zero, this is equivalent to linearly reductive.
\begin{theorem}[Popov]
 Suppose $\mathrm{char}(k)=0$\anton{maybe not needed}. If on every affine variety $X$ with $G$-action, $k[X]^G$ is finitely generated, then $G$ is algebraically reductive.
\end{theorem}
A couple of books I found later: Invariant Theory by Popov and Vinberg, Birkhauser DMV seminar Algebraic Transformation Groups and Invariant Theory, and Lectures on Invariant Theory by Dolgachev.

\subsektion{Nagata's example}

$G=C\cdot G'$ acts on $\AA^n\times \AA^n=\AA^{2n}$ (coordinates $x_i$ and $y_i$). The $G'$ action is given by $(x_k,y_k)\mapsto (x_k+\alpha_ky_k,y_k)$ for $\sum a_{ij}\alpha_j=0$ and $i=1,2,3$, $\alpha_i\in k$. The $C$ action is $c\cdot (x_i,y_i)=(c_ix_i,c_iy_i)$ where $c_1\cdots c_n=1$ and $c_i\in k^\times$.

In some basis, this is block diagonal with blocks $\matx{c_i& \alpha_i\\ 0 & c_i}$, with determinant 1. This group has $\dim G=2n-4$. Take $n=9$. For a suitable choice of $a_{ij}$, we don't get a finitely generated ring of invariants $k[x,y]^G$.

\underline{Step 1}. The $a_{ij}$ are a $3\times 9$ matrix. Suppose $\det (a_{ij})_{i,j=1,2,3}\neq 0$ (the first minor is non-singular). Let $z_i = \sum_{j=1}^n a_{ij} (x_jt/y_j)$ where $t=y_1\cdots y_n$ for $i=1,2,3$.

$k(x,y)^G=k(t,z_1,z_2,z_3)$. Checking that they are invariant uses the calculation $g(x_jt/y_j)=x_j/y_j+\alpha_j$.

$k(x_1,\dots, x_n,y_1,\dots, y_n)=k(z_1,z_2,z_3,x_4,\dots, x_n,y_1,\dots, y_n) = k(t,z_1,z_2,z_3,x_4,\dots, x_n,y_1,\dots, y_{n-1})$. If I pick $H\subseteq G$ such that $\alpha_5=\cdots \alpha_n=0$, we get that $x_4$ is not invariant, and that's the only thing $H$ acts on, so we can eliminate the $x_4$. Procede inductively to get
\[
 k(t,z_1,z_2,z_3,x_4,\dots, x_n,y_1,\dots, y_{n-1} = \cdots = k(t,z_1,z_2,z_3)
\]

The nine columns of $(a_{ij})$ can be regarded as nine points $p_1,\dots, p_9$ in $\PP^2$. Let $R_m$ be the set of homogeneous polynomials $f(z_1,z_2,z_3)$ which have multiplicity at least $m$ at each $p_i$.

\underline{Step 2}. We have that $k[x,y]^G = \{\sum f_m(z_1,z_2,z_3)t^{-1}|f_m\in R_m\}$.

$k[x,y]^G = k[z_1,z_2,z_3,t,t^{-1}]\cap k[x,y]$. I claim you can only invert $t$ and that the $z_i$ don't get inverted. This is because $k[x_1,\dots, x_n,y_1^{\pm 1},\dots, y_n^{\pm 1}]=k[z_1,z_2,z_3,x4,\dots, x_n,y_1^{\pm 1},\dots, y_n^{\pm 1}]$, and then we just impose the condition that it's actually a polynomial in the $y_i$.

$f=\sum f_{i_1,i_2,i_3,m} z_1^{i_1}z_2^{i_2}z_3^{i_3}t^{-m}$. Each $z_i = \sum a_{ij}x_j y_1\dots \hat y_j\dots y_n$. When I divided by $t$, everything is okay except $y_j$.

Multiplicity condition follows from the fact that all $y_i$ must be in non-negative powers.

\underline{Step 3}. Cubic curve $C$ comes into the picture. A cubic curve is an abelian group (depending on a choice of zero, which we choose to be an inflection point).
\begin{claim}
 The order of $p_1+\cdots +p_9$ is $m$ if and only if there exists a homogeneous polynomial $f(z_1,z_2,z_3)$ non-zero on $C$ which has multiplicity $m$ at each $p_i$.
\end{claim}
\begin{proof}
 Let $p+q+r=0$ on $C$. In the group of divisors, we have $[p]+[q]-[r]-[0]$ is the divisor of a rational function. This is equivalent to saying that $r+p+q=0$. Then $m[p_1]+m[p_2]+\cdots + m[p_9]-9m[0]$ is the divisor of a rational function. The numerator of this rational function must have zeros of multiplicity $m$ at each $p_i$. The rational function is $F/\ell^{3m}$, where $\ell$ is the equation of the tangent line at 0.
\end{proof}
We're going to choose the $p_i$ such that $p_1+\cdots +p_9$ does not have finite order.

Consider $f=\sum f_m(z_1,z_2,z_3)t^{-m}$ as a polynomial in $x$ and $y$. If $\deg f=d$, then $\deg f_m=k$ and I have the condition $nk-mn=d$ because each $z_i$ has degree $n$ in $x$ and $y$. So $k=m+d/n$ (and $n=9$). So we have a double grading, and we're going to calculate the degree of each element in the double grading.

Let $R_{k,m}=\{f(z_1,z_2,z_3)$ of degree $k$ and multiplicity at each $p_i$ at least $m_i\}$. This is a finite-dimensional space and I can estimate its dimension. It is the dimension of all polynomials of degree $k$ minus the dimension of polynomials with small multiplicities.
\[
 \dim R_{k,m} \ge \frac{1}{2} (k+1)(k+2) - \frac 92 m(m+1)
\]
The last term comes from looking at $f(p_1,z_2,z_3)$, which is a polynomial in two variables of degree less than $m$, and there are 9 such conditions.
\[
 = \frac 12 (k-3m)(k+3m+3).
\]
This $R_{k.m}$ is a piece in the invariant ring. The dimension is positive if $k>3m$. On the line $k=3m$, I have dimension 1. The polynomial must divide the polynomial which defines my cubic (suppose it's given by $h(z_1,z_2,z_3)=0$), so we can decrease the degree, which shows that we have dimension is exactly 1 on that line. When I draw a parallel line, the dimension increases along the line just from the formula. Now we can see that there cannot be finitely many generators. If there were, they would correspond to some points in the picture.


